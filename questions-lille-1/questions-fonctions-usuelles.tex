\qcmtitle{Fonctions usuelles}

\qcmauthor{Arnaud Bodin, Abdellah Hanani, Mohamed Mzari}



%%%%%%%%%%%%%%%%%%%%%%%%%%%%%%%%%%%%%%%%%%%%%%%%%%%%%%%%%%%%
\section{Fonctions usuelles | 126}




\qcmlink[cours]{http://exo7.emath.fr/cours/ch_usuelles.pdf}{Fonctions usuelles}

\qcmlink[video]{http://youtu.be/l8ZaQUjM5h8}{partie 1. Logarithme et exponentielle}

\qcmlink[video]{http://youtu.be/lGfC0R_FGaM}{partie 2. Fonctions circulaires inverses}

\qcmlink[video]{http://youtu.be/tsN8Wn8j-1U}{partie 3. Fonctions hyperboliques et hyperboliques inverses}

\qcmlink[exercices]{http://exo7.emath.fr/ficpdf/fic00014.pdf}{Fonctions circulaires et hyperboliques inverses}


%-------------------------------
\subsection{Fonctions usuelles | Facile | 126.00}


\subsubsection{Domaine de définition}
 
\begin{question} 
\qtags{motcle=domaine de définition}

Soit $f(x)= \frac{x^2+3x+2}{x^2-2x-1}$ et $ g(x)= \sqrt{x^2-1}$. On notera $D_f$ et $D_g$ le domaine de définition de $f$ et  de $g$ respectivement. Quelles sont les assertions vraies ?
\begin{answers}
    \bad{$D_f=]1-\sqrt 2, 1+\sqrt 2[$}

    \good{$D_f=\Rr \backslash \{ 1-\sqrt 2, 1+\sqrt 2\}$}

    \bad{$D_g=[-1,1]$}

    \good{$D_g=]-\infty, -1]\cup [1, +\infty[$}
\end{answers}
\begin{explanations}
$f$ est définie si et seulement si $x^2-2x-1 \neq 0$, c'est-à-dire $x\neq 1-\sqrt 2$ et $x\neq 1+\sqrt 2$. 
$g$ est définie si et seulement si $x^2-1 \ge 0$, c'est-à-dire $x\ge 1$ ou $x\le -1$.
\end{explanations}

\end{question}


\begin{question}
\qtags{motcle=domaine de définition}

Soit $ f(x)= \sqrt{\frac{1-x}{2-x}} $ et $g(x)=\frac{\sqrt{1-x}}{\sqrt{2-x}}$. On notera $D_f$ et $D_g$ le domaine de définition de $f$ et $g$ respectivement. Quelles sont les assertions vraies ?
\begin{answers}
    \good{$D_f=]-\infty, 1] \cup ]2,+\infty[$}

    \bad{$D_f= [1,2[$}

    \good{$D_g=]-\infty, 1]$}

    \bad{$D_g=]-\infty, 2[$}
\end{answers}
\begin{explanations}
$f$ est définie  si $x\neq 2$ et $\frac{1-x}{2-x}\ge 0$. On déduit que 
$D_f=]-\infty, 1] \cup ]2,+\infty[$. $g$ est définie  si $1-x \ge 0$ et $2-x > 0$,  c'est-à-dire $x\le 1$. 
\end{explanations}

\end{question}







\begin{question} 
\qtags{motcle=domaine de définition}

Soit $ f(x)= \ln(\frac{2+x}{2-x}) $ et $g(x)=x^x$. On notera $D_f$ et $D_g$ le domaine de définition des fonctions $f$ et $g$ respectivement. Quelles sont les assertions vraies ?
\begin{answers}
    \bad{$D_f=\Rr\backslash\{2\}$}

    \good{$D_f=]-2,2[$}

    \bad{$D_g=\Rr$}

    \good{$D_g=]0,+\infty[$}
\end{answers}
\begin{explanations}
$f$ est définie  si $x\neq 2$ et $\frac{2+x}{2-x}\ge 0$. On déduit que 
 $D_f=]-2,2[$. Par définition, $g(x)=e^{x\ln x}$. Donc $g$ est définie si $x>0$.
\end{explanations}

\end{question}


\subsubsection{Fonctions circulaires réciproques}


\begin{question} 
\qtags{motcle=domaine de définition}

Soit $f(x)= \arcsin (2x), \, g(x)= \arccos (x^2-1) $ et $h(x)= \arctan \sqrt{x}$. On notera $D_f,D_g$ et $D_h$  le domaine de définition de $f, g$ et $h$ respectivement. Quelles sont les assertions vraies ?
\begin{answers}
    \bad{$D_f=[-1,1]$}

     \bad{$D_g=[-1,1]$}

    \good{$D_g=[-\sqrt 2, \sqrt 2]$}
    
    \good{$D_h=[0,+\infty[$}

    
\end{answers}
\begin{explanations}
Les fonctions $x\mapsto \arcsin x$ et  $x \mapsto \arccos x$ sont définies sur $[-1,1]$ et la fonction  $x \mapsto  \arctan x$ est définie sur $\Rr$. On déduit que :
$f$ est définie si $-1\le 2x\le 1$, c'est-à-dire $x \in [-\frac{1}{2}, \frac{1}{2}]$, 
$g$ est définie si $-1 \le x^2-1 \le 1$, c'est-à-dire $x\in [-\sqrt 2, \sqrt 2]$ et  $h$ est définie si $x\ge 0$.
\end{explanations}

\end{question}


\begin{question} 
\qtags{motcle=trigonométrie}

Soit $A=\arcsin (\sin \frac{15\pi}{7})$, $B=\arccos (\cos \frac{21\pi}{11})$ et $C=\arctan (\tan \frac{17\pi}{13})$.  Quelles sont les assertions vraies ?

\begin{answers}
    \bad{$A=\frac{15\pi}{7}$}

    \good{$A=\frac{\pi}{7}$}

    \bad{$B=-\frac{\pi}{11}$}

    \good{$C=\frac{4\pi}{13}$}
\end{answers}
\begin{explanations}
On a : $\arcsin x \in [-\frac{\pi}{2},\frac{\pi}{2}], \forall x \in [-1,1]$,   $\arccos x \in [0,\pi], \forall x\in [-1,1]$ et $\arctan x \in ]-\frac{\pi}{2},\frac{\pi}{2}[, \forall x \in \Rr$. En utilisant la périodicité des fonctions sinus, cosinus et tangente, on obtient : $A=\frac{\pi}{7}$, $B=\frac{\pi}{11}$ et $C=\frac{ 4\pi}{13}$.
\end{explanations}


\end{question}





\begin{question} 
\qtags{motcle=trigonométrie}

Soit $f(x)=  \arcsin (\cos  x)$ et  $ g(x)= \arccos (\sin x) $. Quelles sont les assertions vraies ?
\begin{answers}
    \bad{$f$ est périodique de période $\pi$.}

    \good{$g$ est périodique de période $2\pi$.}

    \good{$f$ est une fonction paire.}

    \bad{$g$ est une fonction impaire.}
\end{answers}
\begin{explanations}
Les fonctions $x\ \mapsto   \sin x$ et $x  \mapsto  \cos  x$ sont périodiques de période $2\pi$, donc $f$ et $g$ sont de période $2\pi$. La fonction $x  \mapsto  \cos  x$  est paire, donc $f$ l'est aussi. La fonction $x  \mapsto  \sin   x$ est impaire, mais la fonction $x  \mapsto   \arccos  x$ n'est ni paire ni impaire, donc $g$ n'est ni paire ni impaire.
\end{explanations}

\end{question}



\subsubsection{Equations}



\begin{question}
\qtags{motcle=domaine de définition}
 
Soit $(E)$ l'équation : $\ln (x^2-1) = \ln (x-1) + \ln 2$. Quelles sont les assertions vraies ?
\begin{answers}
    \bad{$(E)$ est définie sur  $]-\infty, -1[ \cup  ]1,+\infty[$.}

    \good{$(E)$ est définie sur  $]1,+\infty[$.}

    \good{$(E)$ n'admet pas de solution.}

    \bad{$(E)$ admet une unique solution $x=1$.}
\end{answers}
\begin{explanations}
$(E)$ est définie si $x^2-1>0$ et $x-1>0$, c'est-à-dire $x>1$. 
Soit $x>1$, alors $(E) \Leftrightarrow  \ln (x-1) + \ln (x+1)= \ln (x-1) + \ln 2  \Leftrightarrow  \ln (x+1)=  \ln 2 \Leftrightarrow  x=1$.
\end{explanations}

\end{question}


\begin{question} 
\qtags{motcle=domaine de définition}

Soit $(E)$ l'équation : $e^{2x}+e^x-2=0$. Quelles sont les assertions vraies ?

\begin{answers}
    \good{$(E)$ est définie sur $\Rr$.}

    \bad{Le domaine de définition de $(E)$ est $\Rr^+$.}

    \bad{$(E)$ admet deux solutions distinctes.}

    \good{$(E)$ admet une unique solution $x=0$.}
\end{answers}
\begin{explanations}
La fonction exponentielle est définie sur $\Rr$, donc $(E)$ est définie sur $\Rr$. 
En posant $y=e^x$, on se ramène à résoudre l'équation du second degré $y^2+y-2=0$. En résolvant cette équation, on obtient $y=1$ ou $y=-2$. Par conséquent, $x=0$.
\end{explanations}


\end{question}

\subsubsection{Etude de fonctions }

\begin{question} 
\qtags{motcle=étude de fonction}

Soit $f(x)=\sqrt[3]{1-x^3}$. Quelles sont les assertions vraies ?
 %$g(x)=\ln (\frac{x-1}{x+1})$.  

\begin{answers}
    \good{$f$ est définie sur $\Rr$.}

    \bad{$f$ est croissante.}

    \good{$f$ est une bijection de $\Rr$ dans $\Rr$.}

    \good{L'application réciproque de $f$ est $f$.}
\end{answers}
\begin{explanations}
La fonction $x\to \sqrt[3]{x}$ est définie sue $\Rr$, elle est strictement croissante et établit une bijection de $\Rr$ dans $\Rr$. On déduit que $f$ est définie sur $\Rr$, elle est strictement décroissante et établit une bijection de $\Rr$ dans $\Rr$. Soit $y\in \Rr$, on a : $y=\sqrt[3]{1-x^3} \Leftrightarrow 1-x^3=y^3 \Leftrightarrow  x=\sqrt[3]{1-y^3}$. Donc l'application réciproque de $f$ est $f$.
\end{explanations}


\end{question}


\begin{question} 
\qtags{motcle=étude de fonction}

Soit $f(x)=\frac{\ln x}{x}$.  Quelles sont les assertions vraies ?

\begin{answers}
    \good{$f$ est définie sur $]0,+\infty[$.}

    \bad{$f$ est croissante sur $]0,+\infty[$.}

    \good{$f$ est une bijection de $]0,e]$ dans $]-\infty, \frac{1}{e}]$.}

    \good{$f$ est une bijection de $[e,+\infty[$ dans $]0, \frac{1}{e}]$.}
\end{answers}
\begin{explanations}
$f$ est définie sur $]0,+\infty[$. En étudiant les variations de $f$, $f$ est strictement croissante sur $]0,e]$ et strictement décroissante sur $[e,+\infty[$. D'autre part, $f(]0,e])=]-\infty, \frac{1}{e}]$ et $f([e,+\infty[)=]0, \frac{1}{e}] $. On déduit que $f$ établit une bijection de $]0,e]$ dans $]-\infty, \frac{1}{e}]$ et de $[e,+\infty[$ dans $]0, \frac{1}{e}]$.
\end{explanations}


\end{question}






%-------------------------------
\subsection{Fonctions usuelles | Moyen | 126.00}

\subsubsection{Domaine de définition}

\begin{question} 
\qtags{motcle=domaine de définition}

Soit $f(x)= \sqrt[3]{1-x^2}$ et $ g(x)= e^{\frac{1}{x}}\sqrt[4]{1-|x|} $. On notera $D_f$ et $D_g$ le domaine de définition de $f$ et $g$ respectivement. Quelles sont les assertions vraies ?
\begin{answers}
    \good{$D_f=\Rr$}

    \bad{$D_f=[-1,1]$}

    \bad{$D_g= \Rr^*$}

    \good{$D_g=[-1,0[\cup ]0,1]$}
\end{answers}
\begin{explanations}
la fonction $x\to \sqrt[3]{x}$ est définie sur $\Rr$, donc $D_f=\Rr$. La fonction $x\mapsto \sqrt[4]{x}$ est définie sur $[0,+\infty[$. On déduit que  $D_g=[-1,0[\cup ]0,1]$.
\end{explanations}

\end{question}



\subsubsection{Equations - Inéquations}

\begin{question} 
\qtags{motcle=équation}

Soit $(E)$ l'équation : $ 4^x-3^x=3^{x+1}- 2^{2x+1}$. Quelles sont les assertions vraies ?

\begin{answers}
    \good{$(E)$ est définie sur $\Rr$.}

    \good{$(E)$ admet une unique solution $x=1$.}

    \bad{$(E)$ admet deux solutions distinctes.}

    \bad{$(E)$ n'admet pas de solution.}
\end{answers}
\begin{explanations}
On a : $(E) \Leftrightarrow 2^{2x}+ 2^{2x+1} = 3^{x+1}+3^x \Leftrightarrow (1+2)2^{2x}=(1+3)3^{x}  \Leftrightarrow 2^{2x-2} =3^{x-1} \Leftrightarrow (2x-2)\ln 2=(x-1) \ln 3 \Leftrightarrow  x=1$.
\end{explanations}


\end{question}





\begin{question} 
\qtags{motcle=inéquation}

Soit $(E)$ l'inéquation : $ \ln |1+x|-\ln |2x+1| \le \ln 2$. Quelles sont les assertions vraies ?

\begin{answers}
    \bad{Le domaine de définition de $(E)$ est  $]-\frac{1}{2}, +\infty[$.}

    \bad{L'ensemble des solutions de $(E)$ est : $ ]-1,-\frac{3}{5}] \cup ]-\frac{1}{3}, + \infty[$.}

    \bad{L'ensemble des solutions de $(E)$ est $]-\infty, -1[ \cup ]-1,-\frac{3}{5}]  $.}

    \good{L'ensemble des solutions de $(E)$ est : $]-\infty, -1[ \cup ]-1,-\frac{3}{5}] \cup [-\frac{1}{3}, + \infty[$.}
\end{answers}
\begin{explanations}
Soit $x \in \Rr \backslash \{-1, -\frac{1}{2}\}$. $(E) \Leftrightarrow \ln \vert \frac{x+1}{4x+2} \vert \le  0 \Leftrightarrow \, (E') : \, \vert \frac{x+1}{4x+2} \vert \le 1 $. 

 %\Leftrightarrow x\ge -\frac{1}{3} \, \mbox{ou} \, x< -\frac{1}{2} $.
 Si $x>-\frac{1}{2}$, $(E')\Leftrightarrow -4x-2 \le x+1\le 4x+2  \Leftrightarrow x \ge -\frac{1}{3}$.
 
 Si  $x<-\frac{1}{2}$, $(E')\Leftrightarrow -4x-2 \ge x+1 \ge 4x+2  \Leftrightarrow x \le -\frac{3}{5}$.
 
 Par conséquent, l'ensemble des solutions de $(E)$ est $]-\infty, -1[ \cup ]-1,-\frac{3}{5}] \cup [-\frac{1}{3}, + \infty[$. 
\end{explanations}
\end{question}


\begin{question} 
\qtags{motcle=étude de fonction}

Soit   $f(x)= \sin x -x$ et $g(x)= e^x-1-x$  Quelles sont les assertions vraies ?

\begin{answers}
    \bad{$f(x) \le  0, \,  \forall x \in \Rr$}

    \good{$f(x) \le  0, \,   \forall x\ge 0$}

    \good{ $g(x) \ge   0, \,   \forall x\ge 0$}

    \good{$g(x) \ge   0, \,  \forall x \in \Rr$}
\end{answers}
\begin{explanations}
On pourra étudier les variations des fonctions $f$ et $g$. On obtient :  $\sin x \le x, \, \forall x\ge 0$ et $e^x \ge 1+x, \, \forall x\in \Rr$.
\end{explanations}


\end{question}



\subsubsection{Fonctions circulaires réciproques}



\begin{question} 
\qtags{motcle=étude de fonction}

Soit $f(x)=\arcsin x + \arccos x$.  Quelles sont les assertions vraies ?

\begin{answers}
    \good{Le domaine de définition de $f$ est $[-1,1]$.}

    \good{$\forall x\in [-1,1], \, f(x)=\frac{\pi}{2}$}

    \bad{$\forall x\in [-1,1], \, f(x)= x$}

    \good{$f$ est une fonction constante.}
\end{answers}
\begin{explanations}
$f$ est définie sur $[-1,1]$,  dérivable sur $]-1,1[$  et $f'(x)=0$, $\forall x \in ]-1,1[$. Puisque $]-1,1[$ est un intervalle, on déduit que $f$ est constante sur $]-1,1[$ et comme $f$ est continue sur $[-1,1]$, $f$ est constante sur $[-1,1]$. Or $f(0)=\frac{\pi}{2}$, donc $f(x)=\frac{\pi}{2}, \, \forall x \in [-1,1]$.
\end{explanations}


\end{question}


\begin{question} 
\qtags{motcle=étude de fonction}

Soit   $f(x)= \arctan x + \arctan (\frac{1}{x})$.  Quelles sont les assertions vraies ?

\begin{answers}
    \good{Le domaine de définition de $f$ est $\Rr^*$.}
    
    \bad{$f$ est une fonction constante.}

    \bad{$\forall x\in \Rr^*, \, f(x)=\frac{\pi}{2}$}

    \good{$\forall x>0, \, f(x)= \frac{\pi}{2}$ }

    
\end{answers}
\begin{explanations}
$f$ est définie sur $\Rr^*$,  dérivable sur  $\Rr^*$ et $f'(x)=0$, $\forall x \in \Rr^*$. Puisque $\Rr^* =]-\infty,0[\cup ]0,+\infty[$, $f$ est constante sur chaque intervalle. Or $f(1)=\frac{\pi}{2}$ et $f(-1)=-\frac{\pi}{2}$, on déduit que $f(x)=\left\{\begin{array}{cc}\frac{\pi}{2},& \mbox{si} \, \, x >0 \\ -\frac{\pi}{2},& \mbox{si} \,  x <0  \end{array}\right.$.
\end{explanations}

\end{question}

\subsubsection{Etude de fonctions}


\begin{question} 
\qtags{motcle=graphe}

Soit $ f(x)= \frac{2x+1}{x-1}$. Quelles sont les assertions vraies ?
\begin{answers}
    \good{$y=2$ est une asymptote à la courbe de $f$ en $+\infty$.}

    \good{La courbe de $f$ admet une asymptote verticale $(x=1)$.}

    \bad{Le point de coordonnées $(1,1)$ est un centre de symétrie du graphe de $f$.}

    \good{Le point de coordonnées $(1,2)$ est un centre de symétrie du graphe de $f$.}
\end{answers}
\begin{explanations}
$\lim_{x\to +\infty}f(x)=2$, donc $y=2$ est une asymptote à la courbe de $f$ en $+\infty$.

$\lim_{x\to 1^+}f(x)=+\infty$ et $\lim_{x\to 1^-}f(x)=-\infty$, donc la droite $x=1$ est une asymptote (verticale) à la courbe de $f$ en $1$.

Le graphe de $f$ admet un centre de symétrie d'abscisse $1$ si et seulement si la fonction $x\to f(1+x)+f(1-x)$ est constante, pour tout $x\neq 0$. Ce qui revient à ce que la fonction $x\to f(x)+f(2-x)$ soit constante, pour tout $x\neq 1$. Or $f(x)+f(2-x)=4=2\times 2, \, \forall x \neq 1$. Donc le point de coordonnées $(1,2)$ est un centre de symétrie du graphe de $f$.
\end{explanations}

\end{question}


\begin{question} 
\qtags{motcle=étude de fonction}

Soit $f(x)= (-1)^{E(x)}$, où  $E(x)$ est la partie entière de $x$. Quelles sont les assertions vraies ?
\begin{answers}
    \bad{$f$ est périodique de période $1$.}

    \good{$f$ est périodique de période $2$.}

    \bad{$f$ est une fonction paire.}

    \good{$f$ est bornée.}
\end{answers}
\begin{explanations}
$|f(x)|=1$, donc $f$ est bornée.

$\forall x \in \Rr, n\in \Zz, \, E(x+n)= E(x)+n$. On déduit que  $f(x+1)=-f(x)$ et $f(x+2)=f(x)$, donc $f$ est de période $2$.

On a : $ \forall x \in \Rr, \, E(x)\le x< E(x)+1$,  donc $\forall x \in \Rr, \, -E(x)-1< -x\le - E(x)$, on déduit que : $E(-x)=\left\{\begin{array}{cc}-x,& \mbox{si} \, \, x \in \Zz \\ -E(x)-1,& \mbox{si} \,  x \notin \Zz  \end{array}\right.$. Il en découle :  $f(-x)=\left\{\begin{array}{cc}f(x),& \mbox{si} \, \, x \in \Zz \\ -f(x),& \mbox{si} \,  x \notin \Zz  \end{array}\right.$. Donc $f$ n'est ni paire ni impaire.
\end{explanations}

\end{question}


\begin{question} 
\qtags{motcle=graphe}

Soit $f(x)=\sqrt{\frac{x^3}{x-1}}$.  Quelles sont les assertions vraies ?

\begin{answers}
    \good{Le domaine de définition de $f$ est $]-\infty, 0]\cup ]1,+\infty[$.}
         
    \bad{$y=x-\frac{1}{2}$ est une asymptote à la courbe de $f$ en $+\infty$.}

    \good{$y=x+\frac{1}{2}$ est une asymptote à la courbe de $f$ en $+\infty$.}

    \good{$y=-x-\frac{1}{2}$ est une asymptote en $-\infty$.}

  
\end{answers}
\begin{explanations}
$f$ est définie si $x\neq 1$ et $\frac{x}{x-1}\ge 0$. On déduit que le domaine de définition de $f$ est $]-\infty, 0]\cup ]1,+\infty[$.

$\lim_{x \to \pm \infty}f(x) = + \infty$,  $\lim_{x \to + \infty}\frac{f(x)}{x} = 1$, $\lim_{x \to + \infty}(f(x)-x) = \lim_{x \to + \infty}x(\sqrt{\frac{x}{x-1}}-1)= \lim_{x \to + \infty} \frac{x}{(x-1)(\sqrt{\frac{x}{x-1}}+1)}= \frac{1}{2}$. Donc $y=x+\frac{1}{2}$ est une asymptote à la courbe de $f$ en $+\infty$.

$\lim_{x \to - \infty}\frac{f(x)}{x} = -1$,  $\lim_{x \to - \infty}(f(x)+x) = \lim_{x \to - \infty}x(1-\sqrt{\frac{x}{x-1}})= -\frac{1}{2}$. Donc $y=-x-  \frac{1}{2}$ est une asymptote à la courbe de $f$ en $-\infty$.
\end{explanations}


\end{question}




\begin{question} 
\qtags{motcle=étude de fonction}

Soit $f(x)=x+ \sqrt{ 1-x^2}$.  Quelles sont les assertions vraies ?

\begin{answers}
    \good{Le domaine de définition de $f$ est $[-1,1]$.}

    \bad{$f$ est croissante sur $[-1,1]$.}

    \bad{$f$ établit une bijection de $[0,1]$ dans $[1,\sqrt 2]$.}

    \good{$f$ établit une bijection de $[-1,\frac{1}{\sqrt 2}]$ dans $[-1,\sqrt 2]$.}
\end{answers}
\begin{explanations}
$f$ est définie sur $[-1,1]$, dérivable sur $]-1,1[$ et $f'(x)= 1-\frac{x}{\sqrt{1-x^2}}$. En étudiant les variations de $f$, on déduit que $f$ établit une bijection de $[-1,\frac{1}{\sqrt 2}]$ dans $[-1,\sqrt 2]$ et de $[\frac{1}{\sqrt 2}, 1]$  dans $[1,\sqrt 2]$.
\end{explanations}


\end{question}




%-------------------------------
\subsection{Fonctions usuelles | Difficile | 126.00}

\subsubsection{Equations}

\begin{question} 
\qtags{motcle=étude de fonction}

Soit $(E)$ l'équation : $ x^x=(\sqrt x)^{x+1}$.  Quelles sont les assertions vraies ?

\begin{answers}

     \good{Le domaine de définition de $(E)$ est $]0,+\infty[$.}
     
    \bad{$(E)$ n'admet pas de solution.}

    \bad{$(E)$ admet deux solutions distinctes.}

    \good{$(E)$ admet une unique solution.}
\end{answers}
\begin{explanations}
$(E)$ est définie si $x>0$. 

Soit $x>0$, alors $(E) \Leftrightarrow x \ln x  = \frac{1}{2}(x+1)\ln x  \Leftrightarrow (x-1) \ln x = 0 \Leftrightarrow x=1.$
\end{explanations}


\end{question}




\begin{question} 
\qtags{motcle=équation}

Soit $(S)$ le système d'équations : $\left\{\begin{array}{ccl}2^x&=&y^2\\2^{x+1}&=&y^{2+x} \end{array}\right.$.  On note $E$ l'ensemble des $(x,y)$ qui vérifient $(S)$. Quelles sont les assertions vraies ?

\begin{answers}
    \bad {$(S)$ est défini  pour tout $ (x,y) \in \Rr ^2$.}

    \bad{Le cardinal de $E$ est $1$.}

    \good{Le cardinal de $E$ est $2$.}
    
    \bad{Le cardinal de $E$ est $4$.}
\end{answers}
\begin{explanations}
$(S)$ est défini pour tout $x\in \Rr$ et $y>0$.

Soit $x\in \Rr$ et $y>0$, $(S) \Leftrightarrow \left\{\begin{array}{ccl}y&=&2^{\frac{x}{2}} \\2^{x+1}&=&2^{x+\frac{x^2}{2}} \end{array}\right.  \Leftrightarrow \left\{\begin{array}{ccl}y&=&2^{\frac{x}{2}} \\x^2&=&2 \end{array}\right. $. 

Donc $E = \{ (\sqrt 2, \sqrt 2^{\sqrt 2})\,  ; (-\sqrt 2, \sqrt 2^{-\sqrt 2})\}$ et donc le cardinal de $E$ est $2$.
\end{explanations}

\end{question}




\begin{question} 
\qtags{motcle=équation}

Soit $(E)$ l'équation : $ \cos 2x = \sin x $. on note $\cal{S}$ l'ensemble des solutions de $(E)$. Quelles sont les assertions vraies ?

\begin{answers}
    \good{Le domaine de définition de $(E)$  est  $\Rr$.}

    \bad{${\cal{S}} = \{\frac{\pi}{6}  +\frac{ 2k\pi}{3}, \, k\in \Zz \}  $}

    \bad{${\cal{S}} =  \{ -\frac{\pi}{2} +2k\pi, \,  k\in \Zz \} $}

    \bad{ ${\cal{S}} =  \{ -\frac{\pi}{2} +k\pi, \,  k\in \Zz \}$}
\end{answers}
\begin{explanations}
$(E) \Leftrightarrow  \cos 2x =  \cos ( \frac{\pi}{2} -x) \Leftrightarrow      \exists k\in \Zz; \,  2x =  \frac{\pi}{2} -x +2k\pi \, \mbox {ou} \, 2x =  -\frac{\pi}{2} +x +2k\pi  \Leftrightarrow      \exists k\in \Zz;\,  x =  \frac{\pi}{6}  +\frac{ 2k\pi}{3} \, \mbox {ou} \, x =  -\frac{\pi}{2} +2k\pi$. 

Donc ${\cal{S}} = \{\frac{\pi}{6}  +\frac{ 2k\pi}{3}, \, k\in \Zz \} \cup \{ -\frac{\pi}{2} +2k\pi, \,  k\in \Zz \} $.

\end{explanations}


\end{question}


\subsubsection{Fonctions circulaires réciproques}

\begin{question} 
\qtags{motcle=équation}

Soit $f$ une fonction définie par l'équation $(E)$ : $ \arcsin f(x) + \arcsin x = \frac{\pi}{2}$. on notera $D_f$ le domaine de définition de $f$. Quelles sont les assertions vraies ?
\begin{answers}
    \bad{$D_f= [-1,1]$}

    \bad{$\forall x \in [-1,1], \, f(x)= -\sqrt{1-x^2}$}

     \good{$\forall x \in [0,1], \, f(x)= \sqrt{1-x^2}$}

    \good{$f$ est une bijection de $[0,1]$ dans $[0,1]$.}
\end{answers}
\begin{explanations}

\end{explanations}
la fonction $x \to \arcsin x$ est définie sur $[-1,1]$ et prend ses valeurs dans $[-\frac{\pi}{2}, \frac{\pi}{2}]$. Si $-1\le x <0$, $-\frac{\pi}{2}  \le  \arcsin x <0$ et si $0\le  x \le 1$, $0\le   \arcsin x \le  \frac{\pi}{2}$. On déduit que $f$ n'est pas définie si  $-1\le x <0$.

Soit $x \in [0,1]$, on a : $(E) \Rightarrow f(x)= \sin (\frac{\pi}{2} - \arcsin x) = \cos (\arcsin x) = \pm \sqrt{1-x^2} $. Or $\arcsin x \in [-\frac{\pi}{2}, \frac{\pi}{2}] $ et la fonction cosinus est positive sur $[-\frac{\pi}{2}, \frac{\pi}{2}] $, donc $(E) \Rightarrow f(x)= \sqrt{1-x^2}$. 

Réciproquement, on considère la fonction $g$ définie sur $[0,1]$ par : $g(x) =  \arcsin  \sqrt{1-x^2} + \arcsin x$. $g$ est dérivable sur $]0,1[$ et $g'(x)= 0$. Comme $g$ est continue sur $[0,1]$,  $g$ est constante sur cet intervalle et en identifiant en $0$, on obtient $g(x)= \frac{\pi}{2},$ pour tout $x \in [0,1]$. On déduit que $f(x) = \sqrt{1-x^2}$, pour tout $x \in [0,1]$. 

Soit $x, y \in [0,1]$, on a : $y=\sqrt{1-x^2} \Leftrightarrow x = \sqrt{1-y^2}$. Donc $f$ est une bijection de $[0,1]$ dans $[0,1]$ et $f^{-1} = f$.
\end{question}





\begin{question} 
\qtags{motcle=trigonométrie}

Soit $f(x)= \arcsin (\frac{2x}{1+x^2})$. On notera $D_f$ le domaine de définition de $f$. Quelles sont les assertions vraies ?
\begin{answers}
    \bad{$D_f=[-1,1]$}
    
    \good{$D_f=\Rr$}

    \good{Si $x \in [-1,1] $, $f(x) = 2 \arctan x$}

    \good{Si $x \ge 1 $, $f(x) = -2 \arctan x+\pi$}

  
\end{answers}
\begin{explanations}
$f$ est définie si $-1\le \frac{2x}{1+x^2} \le 1 $, c'est-à-dire $-1-x^2 \le 2x \le 1+x^2$, ce qui revient à : $(x+1)^2\ge 0$ et $(x-1)^2\ge 0$, ce qui est le cas pour tout $x \in \Rr$.

$f$ est dérivable sur $\Rr \backslash \{-1,1\}$ et 
$f'(x) = \frac{2(1-x^2)}{|1-x^2|(1+x^2)} = \left\{\begin{array}{cc}\frac{2}{1+x^2},& \mbox{si} \, -1 < x < 1 \\ -\frac{2}{1+x^2},& \mbox{si} \, x > 1 \, \mbox{ou} \, x < -1 \end{array}\right. $. Comme $f$ est continue sur $\Rr$, on déduit que : 
$f(x)= \left\{\begin{array}{ccc}2\arctan x + c_1, & \mbox{si} \, -1\le x\le 1 \\ -2\arctan x + c_2, & \mbox{si} \, x\le -1 \\ -2\arctan x + c_3, & \mbox{si} \, x\ge 1 \end{array}\right.$, où $c_1, c_2$ et $c_3$ sont des constantes.
En identifiant en $0$, en $-\infty$ et en $+\infty$, on obtient : 
$f(x)= \left\{\begin{array}{ccc}2\arctan x, & \mbox{si} \, -1\le x\le 1 \\ -2\arctan x -\pi, & \mbox{si} \, x\le -1 \\ -2\arctan x +\pi, & \mbox{si} \, x\ge 1 \end{array}\right.$.
\end{explanations}

\end{question}


\begin{question} 
\qtags{motcle=trigonométrie}

Soit $f(x)= \arccos (\frac{1-x^2}{1+x^2})$. On notera $D_f$ le domaine de définition de $f$. Quelles sont les assertions vraies ?
\begin{answers}
    \good{$D_f= \Rr$}

    \bad{$D_f=[-1,1]$}

    \good{$ f(x)= 2\arctan x+2\pi, \, \forall x\le 0$}

    \good{$f(x)  = -2\arctan |x|+2\pi, \, \forall x \in \Rr$}
\end{answers}
\begin{explanations}
$f$ est définie si $-1\le \frac{1-x^2}{1+x^2} \le 1 $, ce qui est le cas pour tout $x \in \Rr$.

$f$ est dérivable sur $\Rr^*$ et $f'(x) = -\frac{2x}{|x|(1+x^2)} =\left\{\begin{array}{cc}-\frac{2}{1+x^2},& \mbox{si} \, x>0 \\ \frac{2}{1+x^2},& \mbox{si} \, x<0 \end{array}\right. $. Comme $f$ est continue sur $\Rr$, on déduit que : 
$f(x)= \left\{\begin{array}{cc}-2\arctan x + c_1, & \mbox{si} \, x\ge 0 \\ 2\arctan x + c_2, & \mbox{si} \, x\le 0 \end{array}\right. $, où $c_1$ et $ c_2$  sont des constantes.
En identifiant en $+\infty$ et en $-\infty$, on obtient : 
$f(x)= \left\{\begin{array}{cc}-2\arctan x+2\pi, & \mbox{si} \, x\ge 0 \\ 2\arctan x +2\pi, & \mbox{si} \, x\le 0 \end{array}\right.$.  Donc $f(x)  = -2\arctan |x|+2\pi, \, \forall x \in \Rr$.

\end{explanations}

\end{question}




\subsubsection{Etude de foncions}


\begin{question} 
\qtags{motcle=trigonométrie}

Soit $f(x)= \arcsin (\sin x) + \arccos (\cos x)$. On notera $D_f$ le domaine de définition de $f$.  Quelles sont les assertions vraies ?
\begin{answers}
    \bad{$D_f=[-1,1]$}

     \bad{$\forall x \in \Rr, \, f(x)=2x$}

      \good{ $\forall x \in [-\frac{\pi}{2}, 0], \,  f(x) = 0$}
      
     \good{$\forall x \in [-\pi, -\frac{\pi}{2}], \,  f(x) = -\pi -2x$ }
     
\end{answers}
\begin{explanations}
$D_f=\Rr$. $f$ est périodique de période $2\pi$. En simplifiant $f$ sur $[-\pi, \pi]$, on obtient : 
$f(x) = \left\{\begin{array}{cccc}2x,& \mbox{si} \, 0\le x \le \frac{\pi}{2} \\ \pi ,& \mbox{si} \,\frac{\pi}{2} \le x \le \pi \\  0 ,& \mbox{si} \,-\frac{\pi}{2} \le x \le 0 \\ -\pi-2x ,& \mbox{si} \, -\pi \le x \le - \frac{\pi}{2} \end{array}\right.$.

\end{explanations}

\end{question}




\begin{question} 
\qtags{motcle=étude de fonction}

Soit $f(x)= \exp ( \frac{\ln^2 |x|}{\ln^2 |x|+1})$. On notera $D_f$ le domaine de définition de $f$.  Quelles sont les assertions vraies ?
\begin{answers}
    \bad{$D_f=]0,+\infty[$}

     \good{$f$ est paire.}


     \bad{$f$ est croissante sur $]0,+\infty[$.}
     
    \good{$f$ est une bijection de $]0,1]$ dans $[1,e[$.}

\end{answers}
\begin{explanations}
$D_f= \Rr^*$. Pour tout $x\in \Rr^*$, $f(-x)=f(x)$, donc $f$ est paire.

$f$ est dérivable sur $\Rr^*$. Si $x>0$,  $f'(x)= \frac{2\ln x}{x(\ln^2x+1)^2}f(x)$. En étudiant les variations de $f$, on déduit que $f$ est n'est pas monotone sur $]0,+\infty[$ et qu'elle établit une bijection de $]0,1]$ dans $[1,e[$ et de $[1,+\infty[$ dans $[1,e[$.

\end{explanations}

\end{question}




\begin{question} 
\qtags{motcle=étude de fonction}

Soit $f(x)= x^x(1-x)^{1-x}$. On notera $D_f$ le domaine de définition de $f$.  Quelles sont les assertions vraies ?
\begin{answers}
    \good{$D_f=]0,1[$}

    \good{L'ensemble des valeurs de $f$ est $[\frac{1}{2},1[$.}

    \bad{$f$ est croissante $]0,1[$.}

    \good{$f$ est une bijection de $[\frac{1}{2},1[ $ dans $[\frac{1}{2},1[$.}
\end{answers}
\begin{explanations}
Par définition, $f(x)= \exp [x\ln x + (1-x)\ln (1-x)]$, on déduit que  $D_f=]0,1[$. $f$ est dérivable sur $]0,1[$ et $f'(x)= \ln (\frac{x}{1-x}) f(x)$. En étudiant les variations de $f$, $f$ n'est pas monotone sur  $]0,1[$. Elle établit une bijection de $]0,\frac{1}{2}]$ dans $[\frac{1}{2},1[$ et de $[\frac{1}{2},1[ $ dans $[\frac{1}{2},1[$.
\end{explanations}

\end{question}



\begin{question} 
\qtags{motcle=étude de fonction}

Soit $f(x)= (1+\frac{1}{x})^x$. Quelles sont les assertions vraies ?
\begin{answers}
    \bad{$D_f=]0,+\infty[$}

    \good{$\forall x >0, 1 < f(x) < e$ }

    \bad{$\forall x >0, f(x) > e$ }

    \good{$\forall x <-1, f(x) > e$}
    
\end{answers}
\begin{explanations}
Par définition, $f(x)= \exp [x\ln (1+\frac{1}{x})]$, on déduit que  $f$ est définie si  $x\in ]-\infty, -1[\cup ]0,+\infty[$.

$f$ est dérivable sur son domaine de définition et $f'(x)= [\ln (1+\frac{1}{x}) - \frac{1}{x+1}] f(x)$. En étudiant les variations de la fonction $x \mapsto  \ln (1+\frac{1}{x}) - \frac{1}{x+1}$, on déduit les variations de $f$. En particulier, $f$ est croissante sur  $]-\infty, -1[$ et sur $]0,+\infty[$,   $f(]-\infty, -1[)= ]e,+\infty[$  et $f(]0, +\infty[)= ]1,e[$. 
\end{explanations}

\end{question}

\qcmtitle{Logique -- Raisonnement}

\qcmauthor{Arnaud Bodin, Abdellah Hanani, Mohamed Mzari}


%%%%%%%%%%%%%%%%%%%%%%%%%%%%%%%%%%%%%%%%%%%%%%%%%%%%%%%%%%%%
\section{Logique -- Raisonnement | 100}

\qcmlink[cours]{http://exo7.emath.fr/cours/ch_logique.pdf}{Logique et raisonnements}

\qcmlink[video]{http://youtu.be/aWSe1fjJHEM}{Logique}

\qcmlink[video]{http://youtu.be/B-I5yZd0Wbk}{Raisonnements}

\qcmlink[exercices]{http://exo7.emath.fr/ficpdf/fic00002.pdf}{Logique, ensembles, raisonnements}

%-------------------------------
\subsection{Logique | Facile | 100.01}


\begin{question}
\qtags{motcle=vrai/faux}

Soit $P$ une assertion vraie et $Q$ une assertion fausse. Quelles sont les assertions vraies ?
\begin{answers}
    \good{$P$ ou $Q$}

    \bad{$P$ et $Q$}

    \bad{non($P$) ou $Q$}

    \good{non($P$ et $Q$)}
\end{answers}
\begin{explanations}
$P$ ou $Q$ est vraie. Comme $P$ et $Q$ est fausse alors non($P$ et $Q$) est vraie.
\end{explanations}

\end{question}


\begin{question}
\qtags{motcle=implication/équivalence}

Par quoi peut-on compléter les pointillés pour avoir les deux assertions vraies ?
$$x\ge 2 \quad \ldots \quad x^2 \ge 4  \qquad \qquad |y| \le 3 \quad \ldots \quad 0 \le y \le 3$$
\begin{answers}
    \bad{$\Longleftarrow$ et $\implies$}

    \bad{$\implies$ et $\implies$}

    \bad{$\Longleftarrow$ et $\implies$} 
    
    \good{$\implies$ et $\Longleftarrow$}
\end{answers}
\begin{explanations}
Si $x\ge 2$ alors $x^2 \ge 4$, la réciproque est fausse.
Si $0 \le y \le 3$ alors $|y| \le 3$, la réciproque est fausse.
\end{explanations}
\end{question}


\begin{question}
\qtags{motcle=pour tout/il existe}

Quelles sont les assertions vraies ?
\begin{answers}    
    \bad{$\forall x \in \Rr \quad x^2-x \ge 0$} 

    \good{$\forall n \in \Nn \quad n^2-n \ge 0$}

    \good{$\forall x \in \Rr \quad |x^3-x| \ge 0$}

    \good{$\forall n \in \Nn \setminus \{0,1\} \quad n^2-3 \ge 0$}
\end{answers}
\begin{explanations}
Attention, $x^2-x$ est négatif pour $x=\frac12$ par exemple !
\end{explanations}
\end{question}


\begin{question}
\qtags{motcle=pour tout/il existe}

Quelles sont les assertions vraies ?
\begin{answers}  
    \good{$\exists x>0 \quad \sqrt{x} = x$} 

    \bad{$\exists x <0 \quad \exp(x) < 0$}

    \bad{$\exists n \in \Nn \quad n^2 = 17$}
 
    \good{$\exists z \in \Cc \quad z^2 = -4$}
\end{answers}
\begin{explanations}
Oui il existe $x>0$ tel que $\sqrt{x} = x$, c'est $x=1$.
\end{explanations}
\end{question}


\begin{question}
\qtags{motcle=pour tout/il existe}

Un groupe de coureurs $C$ chronomètre ses temps : $t(c)$ désigne le temps (en secondes) du coureur $c$.
Dans ce groupe Valentin et Chloé ont réalisé le meilleur temps de $47$ secondes. Tom est déçu car il est arrivé troisième, avec un temps de $55$ secondes.
À partir de ces informations, quelles sont les assertions dont on peut déduire qu'elles sont vraies ?
\begin{answers}
    \good{$\forall c \in C \quad t(c) \ge 47$}

    \bad{$\exists c \in C \quad 47 < t(c) < 55$}

    \good{$\exists c \in C \quad t(c) > 47$}

    \bad{$\forall c \in C \quad t(c) \le 55$} 
\end{answers}
\begin{explanations}
Comme Tom est troisième, il n'existe pas de $c$ tel que $47 < t(c) < 55$. 
\end{explanations}
\end{question}


\begin{question}
\qtags{motcle=pour tout/il existe}

Quelles sont les assertions vraies ?
\begin{answers}
    \bad{La négation de "$\forall x > 0 \quad \ln(x) \le x$" est "$\exists x \le 0 \quad  \ln(x) \le x$".}

    \good{La négation de "$\exists x > 0 \quad \ln(x^2) \neq x$" est "$\forall x > 0 \quad \ln(x^2) = x$".}

    \bad{La négation de "$\forall x \ge 0 \quad \exp(x) \ge x$" est "$\exists x \ge 0 \quad  \exp(x) \le x$".}

    \bad{La négation de "$\exists x > 0 \quad \exp(x) >  x$" est "$\forall x > 0 \quad \exp(x) < x$".}
\end{answers}
\begin{explanations}
La négation de "$\forall x > 0 \quad P(x)$" est "$\exists x > 0 \quad$ non($P(x)$)".
La négation de "$\exists x > 0 \quad P(x)$" est "$\forall x > 0 \quad$ non($P(x)$)".
\end{explanations}
\end{question}

%-------------------------------
\subsection{Logique | Moyen | 100.01}


\begin{question}
\qtags{motcle=vrai/faux}

Soit $P$ une assertion fausse, $Q$ une assertion vraie et $R$ une assertion fausse. Quelles sont les assertions vraies ?
\begin{answers}
    \bad{$Q$ et ($P$ ou $R$)}

    \bad{$P$ ou ($Q$ et $R$)}

    \good{non($P$ et $Q$ et $R$)}

    \good{($P$ ou $Q$) et ($Q$ ou $R$)}
\end{answers}
\begin{explanations}
Il y a 8 possibilités à tester à chaque fois, selon que $P,Q,R$ soient vraies ou fausses.
\end{explanations}
\end{question}


\begin{question}
\qtags{motcle=vrai/faux}

Soient $P$ et $Q$ deux assertions. Quelles sont les assertions toujours vraies (que $P$ et $Q$ soient vraies ou fausses)  ?
\begin{answers}
    \bad{$P$ et non($P$)}

    \good{non($P$) ou $P$}

    \bad{non($Q$) ou $P$}

    \good{($P$ ou $Q$) ou ($P$ ou non($Q$))}
\end{answers}
\begin{explanations}
On appelle une tautologie une assertion toujours vraie. C'est par exemple le cas de "non($P$) ou $P$", si $P$ est vraie, l'assertion est vraie, si $P$ est fausse, l'assertion est encore vraie !
\end{explanations}
\end{question}


\begin{question}
\qtags{motcle=implication/équivalence}

Par quoi peut-on compléter les pointillés pour avoir une assertion vraie ?
$$|x^2| < 5 \quad \ldots \quad -\sqrt{5} < x < \sqrt{5}$$
\begin{answers}
    \good{$\Longleftarrow$}

    \good{$\implies$}

    \good{$\iff$}

    \bad{Aucune des réponses ci-dessus ne convient.} 
\end{answers}
\begin{explanations}
C'est une équivalence, donc en particulier les implications dans les deux sens sont vraies !
\end{explanations}
\end{question}


\begin{question}
\qtags{motcle=implication/équivalence}

À quoi est équivalent $P \implies Q$ ?
\begin{answers}
    \bad{non($P$) ou non($Q$)}

    \bad{non($P$) et non($Q$)}

    \good{non($P$) ou $Q$}

    \bad{$P$ et non($Q$)}
\end{answers}
\begin{explanations}
La définition (à connaître) de "$P \implies Q$" est "non($P$) ou $Q$".
\end{explanations}
\end{question}


\begin{question}
\qtags{motcle=pour tout/il existe}

Soit $f : ]0,+\infty[ \to \Rr$ la fonction définie par $f(x) = \frac{1}{x}$. Quelles sont les assertions vraies ?
\begin{answers}
    \good{$\forall x \in ]0,+\infty[ \quad \exists y \in \Rr \qquad y = f(x)$}

    \bad{$\exists x \in ]0,+\infty[ \quad \forall y \in \Rr \qquad y = f(x)$} 
    
    \good{$\exists x \in ]0,+\infty[ \quad \exists y \in \Rr \qquad y = f(x)$}

    \bad{$\forall x \in ]0,+\infty[ \quad \forall y \in \Rr \qquad y = f(x)$}
\end{answers}
\begin{explanations}
L'ordre des "pour tout" et "il existe" est très important.
\end{explanations}
\end{question}


\begin{question}
\qtags{motcle=pour tout/il existe}

Le disque centré à l'origine de rayon $1$ est défini par 
$$D = \left\{ (x,y) \in \Rr^2 \mid x^2+y^2 \le 1\right\}.$$
Quelles sont les assertions vraies ?
\begin{answers}
    \bad{$\forall x \in [-1,1] \quad \forall y \in [-1,1] \qquad (x,y) \in D$}
    
    \good{$\exists x \in [-1,1] \quad \exists y \in [-1,1] \qquad (x,y) \in D$}

    \good{$\exists x \in [-1,1] \quad \forall y \in [-1,1] \qquad (x,y) \in D$}

    \good{$\forall x \in [-1,1] \quad \exists y \in [-1,1] \qquad (x,y) \in D$} 
\end{answers}
\begin{explanations}
Faire un dessin permet de mieux comprendre la situation !
\end{explanations}
\end{question}



%-------------------------------
\subsection{Logique | Difficile | 100.01}


\begin{question}
\qtags{motcle=vrai/faux}

On définit l'assertion "ou exclusif", noté "xou" en disant que "$P$ xou $Q$" est vraie lorsque $P$ est vraie, ou $Q$ est vraie, mais pas lorsque les deux sont vraies en même temps. Quelles sont les assertions vraies ?
\begin{answers}
    \bad{Si "$P$ ou $Q$" est vraie alors "$P$ xou $Q$" aussi.}

    \good{Si "$P$ ou $Q$" est fausse alors "$P$ xou $Q$" aussi.}
    
    \good{"$P$ xou $Q$" est équivalent à "($P$ ou $Q$) et (non($P$) ou non($Q$))"}

    \bad{"$P$ xou $Q$" est équivalent à "($P$ ou $Q$) ou (non($P$) ou non($Q$))"}
\end{answers}
\begin{explanations}
Commencer par faire la table de vérité de "$P$ ou $Q$".
\end{explanations}
\end{question}


\begin{question}
\qtags{motcle=vrai/faux}

Soient $P$ et $Q$ deux assertions. Quelles sont les assertions toujours vraies (que $P$, $Q$ soient vraies ou fausses)  ?
\begin{answers}
    \good{($P \implies Q$) ou ($Q \implies P$)}

    \good{($P \implies Q$) ou ($P$ et non($Q$))}

    \good{$P$ ou ($P \implies Q$)}

    \bad{($P \iff Q$) ou (non($P$) $\iff$ non($Q$))}
\end{answers}
\begin{explanations}
Tester les quatre possibilités selon que $P,Q$ sont vraies ou fausses.
\end{explanations}
\end{question}


\begin{question}
\qtags{motcle=implication/équivalence}

À quoi est équivalent $P \Longleftarrow Q$ ?
\begin{answers}  
    \good{non($Q$) ou $P$}

    \bad{non($Q$) et $P$}
      
    \bad{non($P$) ou $Q$}

    \bad{non($P$) et $Q$}
\end{answers}
\begin{explanations}
La définition (à connaître) de "$P \implies Q$" est "non($P$) ou $Q$".
\end{explanations}
\end{question}


\begin{question}
\qtags{motcle=pour tout/il existe}

Soit $f : \Rr \to \Rr$ la fonction définie par $f(x)=\exp(x)-1$.
Quelles sont les assertions vraies ?
\begin{answers}
    \good{$\forall x,x' \in \Rr  \qquad x \neq x' \implies f(x) \neq f(x')$}
    \good{$\forall x,x' \in \Rr  \qquad x \neq x' \Longleftarrow f(x) \neq f(x')$}
    \bad{$\forall x,x' \in \Rr  \qquad x \neq x' \implies (\exists y \in \Rr \quad f(x) < y < f(x'))$}
    \good{$\forall x,x' \in \Rr  \qquad  f(x)\times f(x') < 0 \implies x\times x' < 0$}    
\end{answers}
\begin{explanations}
Dessiner le graphe de $f$ pour mieux comprendre ! 
Même si $f(x) \neq f(x')$ cela ne veut pas dire que $f(x) < f(x')$, l'inégalité pourrait être dans l'autre sens.
\end{explanations}
\end{question}


\begin{question}
\qtags{motcle=pour tout/il existe}

On considère l'ensemble 
$$E = \left\{ (x,y) \in \Rr^2 \mid 0 \le x \le 1 \text{ et } y \ge \sqrt{x}  \right\}.$$
Quelles sont les assertions vraies ?
\begin{answers}
    \good{$\forall y \ge 0 \quad \exists x \in [0,1] \qquad (x,y) \in E$}
    
    \good{$\exists y \ge 0 \quad \forall x \in [0,1] \qquad (x,y) \in E$}

    \bad{$\forall x \in [0,1] \quad \exists y \ge 0 \qquad (x,y) \notin E$}

    \bad{$\forall x \in [0,1] \quad \forall y \ge 0 \qquad (x,y) \notin E$} 
\end{answers}
\begin{explanations}
Faire un dessin de l'ensemble $E$.
\end{explanations}
\end{question}


\begin{question}
\qtags{motcle=pour tout/il existe}

Soit $f : ]0,+\infty[ \to ]0,+\infty[$ une fonction.
Quelles sont les assertions vraies ?
\begin{answers}
    \bad{La négation de "$\forall x > 0 \quad \exists y > 0 \quad y \neq f(x)$" est "$\exists x > 0 \quad \exists y > 0 \quad y = f(x)$".}
    
    \bad{La négation de "$\exists x > 0 \quad \forall y > 0 \quad y \times f(x)>0$" est "$\forall x > 0 \quad \exists y > 0 \quad y\times f(x) < 0$".}
    
    \bad{La négation de "$\forall x,x' > 0 \quad x \neq x' \implies f(x) \neq f(x')$" est "$\exists x,x' > 0 \quad x = x'$ et $f(x) = f(x')$".}

    \good{La négation de "$\forall x,x' > 0 \quad f(x) = f(x') \implies x = x'$" est "$\exists x,x' > 0 \quad x \neq x'$ et $f(x) = f(x')$".}
\end{answers}
\begin{explanations}
La négation du "$\forall x > 0 \quad \exists y > 0 \ldots$" commence par "$\exists x > 0 \quad \forall y > 0$.
La négation de "$f(x) = f(x') \implies x = x'$" est "$f(x) = f(x')$ et $x \neq x'$".
\end{explanations}
\end{question}



%-------------------------------
\subsection{Raisonnement | Facile | 100.03, 100.04}


\begin{question}
\qtags{motcle=raisonnement}

Je veux montrer que $\frac{n(n+1)}{2}$ est un entier, quelque soit $n\in\Nn$.  Quelles sont les démarches possibles ?
\begin{answers}    
    \bad{Montrer que la fonction $x \mapsto x(x+1)$ est paire.}
    
    \good{Séparer le cas $n$ pair, du cas $n$ impair.}

    \bad{Par l'absurde, supposer que $\frac{n(n+1)}{2}$ est un réel, puis chercher une contradiction.}

    \bad{Le résultat est faux, je cherche un contre-exemple.} 
\end{answers}
\begin{explanations}
Séparer le cas $n$ pair, du cas $n$ impair. Dans le premier cas, on peut écrire $n=2k$ (avec $k\in \Nn$), dans le second cas $n=2k+1$, puis calculer $\frac{n(n+1)}{2}$. 
\end{explanations}
\end{question}


\begin{question}
\qtags{motcle=récurrence}

\qkeeporder

Je veux montrer par récurrence l'assertion $H_n : 2^n > 2n-1$, pour tout entier $n$ assez grand. Quelle étape d'initialisation est valable ?
\begin{answers}
    \bad{Je commence à $n=0$.}

    \bad{Je commence à $n=1$.}

    \good{Je commence à $n=2$.}

    \good{Je commence à $n=3$.} 
\end{answers}
\begin{explanations}
L'initialisation peut commencer à n'importe quel entier $n_0 \ge 2$.
\end{explanations}
\end{question}


\begin{question}
\qtags{motcle=récurrence}

Je veux montrer par récurrence l'assertion $H_n : 2^n > 2n-1$, pour tout entier $n$ assez grand. Pour l'étape d'hérédité je suppose $H_n$ vraie, quelle(s) inégalité(s) dois-je maintenant démontrer ?
\begin{answers}
    \good{$2^{n+1} > 2n+1$}
    
    \bad{$2^{n} > 2n-1$}

    \bad{$2^{n} > 2(n+1)-1$}

    \bad{$2^{n}+1 > 2(n+1)-1$} 
\end{answers}
\begin{explanations}
$H_{n+1}$ s'écrit $2^{n+1} > 2(n+1)-1$, c'est-à-dire $2^{n+1} > 2n+1$.
\end{explanations}
\end{question}


\begin{question}
\qtags{motcle=pour tout/il existe}

Chercher un contre-exemple à une assertion du type 
"$\forall x \in E$ l'assertion $P(x)$ est vraie" revient à prouver l'assertion :
\begin{answers}
    \bad{$\exists! x \in E \quad$ l'assertion $P(x)$ est fausse.}

    \good{$\exists x \in E \quad$ l'assertion $P(x)$ est fausse.}
    
    \bad{$\forall x \notin E \quad$ l'assertion $P(x)$ est fausse.}     
    
    \bad{$\forall x \in E \quad$ l'assertion $P(x)$ est fausse.}
\end{answers}
\begin{explanations}
Un contre-exemple, c'est trouver un $x$ qui ne vérifie pas $P(x)$. (Rien ne dit qu'il est unique.)
\end{explanations}
\end{question}



%-------------------------------
\subsection{Raisonnement | Moyen | 100.03, 100.04}



\begin{question}
\qtags{motcle=implication/équivalence}

J'effectue le raisonnement suivant avec deux fonctions $f,g : \Rr \to \Rr$.
$$\forall x \in \Rr \quad f(x)\times g(x) = 0$$ 
$$\implies \forall x \in \Rr \quad \big(f(x) = 0 \text{ ou } g(x) = 0\big)$$
$$\implies \big(\forall x \in \Rr \quad f(x) = 0\big) \ \text{ ou } \ \big(\forall x \in \Rr \quad g(x) = 0\big)$$
\begin{answers}
    \bad{Ce raisonnement est valide.}

    \bad{Ce raisonnement est faux car la première implication est fausse.}

    \good{Ce raisonnement est faux car la seconde implication est fausse.}

    \bad{Ce raisonnement est faux car la première et la seconde implication sont fausses.}   
\end{answers}
\begin{explanations}
On ne peut pas distribuer un "pour tout" avec un "ou". 
\end{explanations}
\end{question}


\begin{question}
\qtags{motcle=récurrence}

Je souhaite montrer par récurrence une certaine assertion $H_n$, pour tout entier $n\ge0$.
Quels sont les débuts valables pour la rédaction de l'étape d'hérédité ?
\begin{answers}
    \bad{Je suppose $H_n$ vraie pour tout $n\ge0$, et je montre que $H_{n+1}$ est vraie.}

    \bad{Je suppose $H_{n-1}$ vraie pour tout $n\ge1$, et je montre que $H_{n}$ est vraie.}
    
    \good{Je fixe $n\ge0$, je suppose $H_n$ vraie, et je montre que $H_{n+1}$ est vraie.} 
    
    \bad{Je fixe $n\ge0$ et je montre que $H_{n+1}$ est vraie.}
\end{answers}
\begin{explanations}
La récurrence a une rédaction très rigide. Sinon on raconte vite n'importe quoi !
\end{explanations}
\end{question}


\begin{question}
\qtags{motcle=récurrence}

Je veux montrer que $e^x > x$ pour tout $x$ réel avec $x \ge 1$.
L'initialisation est vraie pour $x=1$, car $e^1 = 2,718\ldots >1$.
Pour l'hérédité, je suppose $e^x>x$ et je calcule :
$$e^{x+1} = e^x \times e > x  \times e \ge x \times 2 \ge x + 1.$$
Je conclus par le principe de récurrence.
Pour quelles raisons cette preuve n'est pas valide ?
\begin{answers}
    \bad{Car il faudrait commencer l'initialisation à $x=0$.}

    \good{Car $x$ est un réel.}
    
    \bad{Car l'inégalité $e^x > x$ est fausse pour $x\le0$.}

    \bad{Car la suite d'inégalités est fausse.} 
\end{answers}
\begin{explanations}
La récurrence c'est uniquement avec des entiers !
\end{explanations}
\end{question}


\begin{question}
\qtags{motcle=récurrence}

Pour montrer que l'assertion 
"$\forall n \in \Nn \quad n^2 > 3n-1$" est fausse,
quels sont les arguments valables ?
\begin{answers}
    \bad{L'assertion est fausse, car pour $n=0$ l'inégalité est fausse.}

    \good{L'assertion est fausse, car pour $n=1$ l'inégalité est fausse.}
    
    \good{L'assertion est fausse, car pour $n=2$ l'inégalité est fausse.}     
    
    \good{L'assertion est fausse, car pour $n=1$ et $n=2$ l'inégalité est fausse.}
\end{answers}
\begin{explanations}
C'est faux pour $n=1$ et $n=2$, mais bien sûr, un seul cas suffit pour que l'assertion soit fausse. 
\end{explanations}
\end{question}





%-------------------------------
\subsection{Raisonnement | Difficile | 100.03, 100.04}


\begin{question}
\qtags{motcle=implication/équivalence}

Le raisonnement par contraposée est basé
sur le fait que "$P \implies Q$" est équivalent à:
\begin{answers}
    \bad{"non($P$) $\implies$ non($Q$)".}

    \good{"non($Q$) $\implies$ non($P$)".}

    \bad{"non($P$) ou $Q$".}

    \bad{"$P$ ou non($Q$)".} 
\end{answers}
\begin{explanations}
La contraposée de "$P \implies Q$" est "non($Q$) $\implies$ non($P$)".
\end{explanations}
\end{question}


\begin{question}
\qtags{motcle=implication/équivalence}

Par quelle phrase puis-je remplacer la proposition logique "$P \Longleftarrow Q$" ?
\begin{answers}
    \good{"$P$ si $Q$"}

    \bad{"$P$ seulement si $Q$"}

    \bad{"$Q$ est une condition nécessaire pour obtenir $P$"}

    \good{"$Q$ est une condition suffisante pour obtenir $P$"} 
\end{answers}
\begin{explanations}
C'est plus facile si on comprend que "$P \Longleftarrow Q$", c'est "$Q \implies P$", autrement dit "si $Q$ est vraie, alors $P$ est vraie".
\end{explanations}
\end{question}


\begin{question}
\qtags{motcle=implication/équivalence}

Quelles sont les assertions vraies ?
\begin{answers}
    \bad{La négation de "$P \implies Q$" est "non($Q$) ou $P$"}

    \good{La réciproque de "$P \implies Q$" est "$Q \implies P$"}

    \bad{La contraposée de "$P \implies Q$" est "non($P$) $\implies$ non($Q$)"}

    \bad{L'assertion "$P \implies Q$" est équivalente à "non($P$) ou non($Q$)"} 
\end{answers}
\begin{explanations}
Il faut revenir à la définition de "$P \implies Q$" qui est "non($P$) ou $Q$".
\end{explanations}
\end{question}


\begin{question}
\qtags{motcle=raisonnement}

Je veux montrer que $\sqrt{13} \notin \Qq$ par un raisonnement par l'absurde. Quel schéma de raisonnement est adapté ?
\begin{answers}
    \good{Je suppose que $\sqrt{13}$ est rationnel et je cherche une contradiction.}

    \bad{Je suppose que $\sqrt{13}$ est irrationnel et je cherche une contradiction.}

    \bad{J'écris $13 = \frac{p}{q}$ (avec $p,q$ entiers) et je cherche une contradiction.}
    
    \good{J'écris $\sqrt{13} = \frac{p}{q}$ (avec $p,q$ entiers) et je cherche une contradiction.}
\end{answers}
\begin{explanations}
Par l'absurde on suppose que $\sqrt{13} \in \Qq$, c'est-à-dire que c'est un nombre rationnel, autrement dit qu'il s'écrit $\frac{p}{q}$, avec $p$, $q$ entiers. Voir la preuve que $\sqrt{2} \notin \Qq$.
\end{explanations}
\end{question}





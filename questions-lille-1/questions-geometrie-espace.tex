\qcmtitle{Géométrie dans l'espace}

\qcmauthor{Arnaud Bodin, Abdellah Hanani, Mohamed Mzari}

%%%%%%%%%%%%%%%%%%%%%%%%%%%%%%%%%%%%%%%%%%%%%%%%%%%%%%%%%%%%
\section{Géométrie dans l'espace | 141}


\qcmlink[exercices]{http://exo7.emath.fr/ficpdf/fic00159.pdf}{Droites du plan ; droites et plans de l'espace}


Pour ces questions, l'espace est muni d'un repère orthonormé direct  $(O, \vec{i}, \vec{j}, \vec{k})$.

%------------------------------- 
\subsection{Produit scalaire -- Produit vectoriel -- Déterminant | Facile | 141.01}

 
 
\begin{question}
\qtags{motcle=vecteurs}

Soit $\vec{u}(1,1,1), \vec{v}(1,-1,0)$ et $\vec{w}(0,1,1)$ trois vecteurs. Quelles sont les assertions vraies ?
\begin{answers}
    \good{$\vec{u}$ et $\vec{v}$ sont orthogonaux.}

    \bad{$\vec{v}$ et $\vec{w}$ sont colinéaires.}

    \good{$(O,\vec{u},\vec{v},\vec{w})$ est un repère de l'espace.}

    \bad{$(O,\vec{u},\vec{v},\vec{w})$ est un repère orthonormé de l'espace.}
\end{answers}
\begin{explanations}
Deux vecteurs $\vec{u}$ et $\vec{v}$ sont orthogonaux si et seulement si $ \vec{u} \cdot \vec{v}=0$.
 $(O,\vec{u},\vec{v},\vec{w})$ est un repère si et seulement si $\det (\vec{u},\vec{v},\vec{w}) \neq 0$.
\end{explanations}

\end{question}


\begin{question}
\qtags{motcle=vecteurs}
 
Soit $A(1,1,1), B(0,1,1)$ et $C(1,0,1)$ trois points. Quelles sont les assertions vraies ?
\begin{answers}
    \bad{$A$, $B$ et $C$ sont alignés.}

    \bad{$A,B$ et $C$ forment un triangle d'aire $\frac{1}{3}$.}

    \good{$A,B$ et $C$ forment un triangle d'aire $\frac{1}{2}$.}

    \bad{Les vecteurs $\overrightarrow{AB}$ et  $\overrightarrow{AC}$ sont colinéaires.}
\end{answers}
\begin{explanations}
L'aire du triangle $ABC$ est donnée par : $\frac{1}{2} \Vert \overrightarrow{AB} \wedge \overrightarrow{AC} \Vert$. 
\end{explanations}

\end{question}




%------------------------------- 
\subsection{Aire -- Volume | Moyen | 141.02}
 
 
\begin{question} 
\qtags{motcle=aire/volume}

Soit $\vec{u}(-1,1,1), \vec{v}(0,1,2)$ et $\vec{w}(1,0,-1)$ trois vecteurs. Quelles sont les assertions vraies ?
\begin{answers}

      \bad{L'aire du parallélogramme engendré par $\vec{u}$ et  $\vec{v}$ est : $\sqrt 3$.}
      
       \good{L'aire du parallélogramme engendré par $\vec{u}$ et  $\vec{v}$ est : $ \sqrt 6$.}
      
    \bad{Le volume du parallélépipède engendré par $\vec{u}$, $\vec{v}$ et  $\vec{w}$ est $1$.}

    \good{Le volume du parallélépipède engendré par $\vec{u}$, $\vec{v}$ et  $\vec{w}$ est $2$.}

\end{answers}
\begin{explanations}
L'aire du parallélogramme engendré par deux vecteurs $\vec{u}$ et $\vec{v}$ est donnée par : $ \Vert \vec{u} \wedge \vec{v} \Vert$.
Le volume du parallélépipède engendré par trois vecteurs $\vec{u}$, $\vec{v}$ et  $\vec{w}$ est donné par : $|\det (\vec{u}, \vec{v},\vec{w})|$.
\end{explanations}

\end{question}

%-------------------------------
\subsection{Plans | Facile | 141.03}

\begin{question}
\qtags{motcle=équation droite/plan}

Soit $P$ le plan passant par $A(1,1,0)$ et  de vecteur normal $\vec{n}(1,-1,1)$. Quelles sont les assertions vraies ?
\begin{answers}
    \bad{Une équation cartésienne de $P$ est $x-y+z=1$.}

    \good{Une équation cartésienne de $P$ est $x-y+z=0$.}

     \good {Une représentation paramétrique de $P$ est :
 $$\left\{\begin{array}{ccl}x&=&t-s\\y&=&t\\ z&=&s,\quad (t,s\in \Rr) \end{array}\right.$$}

    \good{Une représentation paramétrique de $P$ est :
$$\left\{\begin{array}{ccl}x&=&t\\y&=&s\\ z&=&s-t,\quad (t,s \in \Rr)\end{array}\right.$$}
   
\end{answers}
\begin{explanations}
Une équation de $P$ est de la forme : $x-y+z+a=0$ et on cherche $a$ pour que $A$ appartienne à $P$. On résout cette équation pour trouver une représentation paramétrique. 
\end{explanations}

\end{question}


\begin{question}
\qtags{motcle=équation droite/plan}
 
Soit $P$ le plan passant par $A(-1,1,1)$ et dirigé par les vecteurs
 $\vec{u}(0,1,1)$ et $\vec{v}(1,0,1)$. Quelles sont les assertions vraies ?
\begin{answers}

    \good {Une représentation paramétrique de $P$ est :
 $$\left\{\begin{array}{ccl}x&=&-1+s\\y&=&1+t\\ z&=&1+t+s,\quad (t,s\in \Rr) \end{array}\right.$$}

    \bad {Une représentation paramétrique de $P$ est :
 $$\left\{\begin{array}{ccl}x&=&-1+t\\y&=&1+s\\ z&=&-1+t+s,\quad (t,s\in \Rr)\end{array}\right.$$}
 
   \bad{Une équation cartésienne de $P$ est $x+y+z=-1$.}
    
    \good{Une équation cartésienne de $P$ est $x+y-z=-1$.}
 
\end{answers}
\begin{explanations}
On peut trouver une équation cartésienne, à partir d'une représentation paramétrique,  en éliminant les paramètres.
\end{explanations}

\end{question}


\begin{question} 
\qtags{motcle=équation droite/plan}

Soit $P$ le plan passant par les points $A(0,1,0)$, $B(1,-1,0)$ et $C(0,1,1)$. Quelles sont les assertions vraies ?
\begin{answers}

    \good {Une représentation paramétrique de $P$ est :
 $$\left\{\begin{array}{ccl}x&=&s\\y&=&1-2s\\ z&=&t,\quad (t,s\in \Rr)\end{array}\right.$$}

    \bad {Une représentation paramétrique de $P$ est :
 $$\left\{\begin{array}{ccl}x&=&t\\y&=&s\\ z&=&1+2s,\quad (t,s\in \Rr)\end{array}\right.$$}
 
   \bad{Une équation cartésienne de $P$ est $2x+z=1$.}
    
    \good{Une équation cartésienne de $P$ est $2x+y=1$.}
   
\end{answers}
\begin{explanations}
$P$ est le plan passant par $A$ et dirigé par les vecteurs $\overrightarrow{AB}$ et $\overrightarrow{AC}$.
\end{explanations}

\end{question}

%-------------------------------
\subsection{Droites de l'espace | Facile | 141.04}


\begin{question} 
\qtags{motcle=équation droite/plan}   

Soit $D$ la droite passant par le point $A(2,-1,1)$ et dirigée par le vecteur $\vec{u}(-1,1,0)$. Quelles sont les assertions vraies ?
\begin{answers}

    \good {Une représentation paramétrique de $D$ est :
 $$\left\{\begin{array}{ccl}x&=&2-t\\y&=&-1+t\\ z&=&1,\quad (t\in \Rr) \end{array}\right.$$}

    \bad {Une représentation paramétrique de $D$ est :
 $$\left\{\begin{array}{ccl}x&=&2-t\\y&=&-1+t\\ z&=&-t,\quad (t\in \Rr)\end{array}\right.$$}
 
   \good{Une représentation cartésienne de $D$ est : 
  $$\left\{\begin{array}{ccl}x+y&=&1\\z&=&1 \end{array}\right.$$}
    
    \bad{Une représentation cartésienne de $D$ est : 
  $$\left\{\begin{array}{ccl}x+y&=&0\\z&=&1 \end{array}\right.$$}
    
\end{answers}
\begin{explanations}
On peut trouver une représentation cartésienne, à partir d'une représentation   paramétrique en éliminant le paramètre.
\end{explanations}

\end{question}


\begin{question} 
\qtags{motcle=équation droite/plan}     
   
Soit $D$ la droite passant par le point $A(-1,1,2)$ et perpendiculaire au plan d'équation cartésienne : $x+y+z=1$. Quelles sont les assertions vraies ?
\begin{answers}

    \good {Une représentation paramétrique de $D$ est :
 $$\left\{\begin{array}{ccl}x&=&-1+t\\y&=&1+t\\ z&=&2+t,\quad (t\in \Rr) \end{array}\right.$$}

    \good {Une représentation paramétrique de $D$ est :
 $$\left\{\begin{array}{ccl}x&=&-1-t\\y&=&1-t\\ z&=&2-t,\quad (t\in \Rr) \end{array}\right.$$}
 
   \good{Une représentation cartésienne de $D$ est : 
  $$\left\{\begin{array}{ccl}x-y&=&-2\\y-z&=&-1 \end{array}\right.$$}
    
    \bad{Une représentation cartésienne de $D$ est : 
  $$\left\{\begin{array}{ccl}x+y&=&2\\x+z&=&1 \end{array}\right.$$}
    
\end{answers}
\begin{explanations}
$D$ est dirigée par le vecteur $\vec{u}(1,1,1)$.
\end{explanations}

\end{question}





%-------------------------------
\subsection{Plans -- Droites | Moyen | 141.03, 141.04}

\begin{question}
\qtags{motcle=équation droite/plan}

Soit $a$ et $b$ deux réels, $D$ et $D'$ deux droites de représentations paramétriques :
 $$D: \left\{\begin{array}{ccl}x&=&1+2t\\y&=&t\\ z&=&-1+at,\quad (t\in \Rr) \end{array}\right. D': \left\{\begin{array}{ccl}x&=&-3+bt\\y&=&-t\\ z&=&1+t,\quad (t\in \Rr) \end{array}\right. 
 $$ Quelles sont les assertions vraies ?
\begin{answers}
    \bad{$D$ et $D'$ sont parallèles si et seulement si $a=2$ et $b=3$.}

    \good{$D$ et $D'$ sont parallèles si et seulement si $a=-1$ et $b=-2$.}

    \bad{$D$ et $D'$ sont orthogonales  si et seulement si $a=1$ et $b=0$.}

    \good{$D$ et $D'$ sont orthogonales  si et seulement si $a=1-2b, \, b \in \Rr$.}
   
\end{answers}
\begin{explanations}
Si $D$ est dirigée par un vecteur $\vec{u}$ et $D'$ est dirigée par un vecteur $\vec{v}$, $D$ et $D'$ sont parallèles si et seulement si $\vec{u} \wedge \vec{v} = \overrightarrow 0$. $D$ et $D'$ sont othogonales si et seulement si $\vec{u} \cdot  \vec{v}=0$.
\end{explanations}

\end{question}


\begin{question}
\qtags{motcle=équation droite/plan}

Soit $P : x+y-z=0$,  $P' : x-y=2$ et $P'' : y-z=3$ trois plans. L'intersection de ces trois plans est : 
\begin{answers}

    \bad{Vide.}

    \bad{Une droite.}
 
    \good{Un point.}
    
    \good{Le point de coordonnées $(-3,-5,-8)$.}
   
\end{answers}
\begin{explanations}
On résout le système constitué des équations des trois plans.
\end{explanations}

\end{question}


\begin{question} 
\qtags{motcle=équation droite/plan}

Soit $P : x-y-z=-2$,  $P' : x+z=2$ deux plans et $D$ la droite :
$\left\{\begin{array}{ccl}x&=&1+t\\y&=&2+2t\\ z&=&1-t,\quad (t\in \Rr)\end{array}\right.$ Quelles sont les assertions vraies ?
\begin{answers}

    \good{$D\subset P'$}

    \good{$D=P\cap P'$}
 
    \bad{$D\cap P=\emptyset$}
    
    \bad{$D\cap P'=\emptyset$}
 
\end{answers}
\begin{explanations}
On vérifie que  $D=P\cap P'$.
\end{explanations}

\end{question}


\begin{question} 
\qtags{motcle=équation droite/plan}  
  
Soit $P : x+y-z=1$,  $P' : x+z=-1$ deux plans et $Q$ le plan passant par $A(1,1,1)$ et perpendiculaire à $P$ et à $P'$. Quelles sont les assertions vraies ?
\begin{answers}

    \bad{Une équation cartésienne de $Q$ est $x+2y-z+2=0$.}

    \good{Une équation cartésienne de $Q$ est $x-2y-z+2=0$.}
 
   \bad{Une représentation paramétrique de $Q$ est :
 $$\left\{\begin{array}{ccl}x&=&1-t\\y&=&1+t-s\\ z&=&1+t+s,\quad (t,s \in \Rr) \end{array}\right.$$}
    
    \good{Une représentation paramétrique de $Q$ est :
 $$\left\{\begin{array}{ccl}x&=&1+t+s\\y&=&1+t\\ z&=&1-t+s,\quad (t,s \in \Rr) \end{array}\right.$$}
    
\end{answers}

\begin{explanations}
$Q$ passe par $A$ et est dirigé par $\vec{n}$ et $\vec{n'}$, où $\vec{n}$ et $\vec{n'}$ sont des vecteurs normaux à $P$ et à $P'$ respectivement.
\end{explanations}

\end{question}



\begin{question}
\qtags{motcle=équation droite/plan}
 
On considère la droite $D : \left\{\begin{array}{ccl}x&=&1-t\\y&=&t\\ z&=&-1+2t,\quad (t \in \Rr) \end{array}\right.$ et le plan $P$  passant par $A(0,1,1)$ et perpendiculaire à $D$.  Quelles sont les assertions vraies ?
\begin{answers}

    \good{Une équation cartésienne de $P$ est $x-y-2z+3=0$.}

    \bad{Une équation cartésienne de $P$ est $x-2y-2z+2=0$.}
 
   \good{Une représentation paramétrique de $P$ est :
 $$\left\{\begin{array}{ccl}x&=&t\\y&=&3+t-2s\\ z&=&s,\quad (t,s \in \Rr) \end{array}\right.$$}
    
    \bad{Une représentation paramétrique de $P$ est :
 $$\left\{\begin{array}{ccl}x&=&t\\y&=&2+t-2s\\ z&=&s,\quad (t,s \in \Rr) \end{array}\right.$$}
  
\end{answers}
\begin{explanations}
Un vecteur directeur de $D$ est un vecteur normal à $P$.
\end{explanations}

\end{question}


\begin{question}
\qtags{motcle=équation droite/plan}
 
On considère les deux plans  $P : \left\{\begin{array}{ccl}x&=&1+t+s\\y&=&-1+t\\ z&=&2+t-s,\quad (t,s  \in \Rr)  \end{array}\right.$ et 
$P': \left\{\begin{array}{ccl}x&=&3+2t\\y&=&t+s\\ z&=&2+2s,\quad (t,s  \in \Rr) \end{array}\right.$  Quelles sont les assertions vraies ?
\begin{answers}

    \bad{$P\cap P'$ est une droite.}

    \bad{$P$ et $P'$ sont perpendiculaires.}
 
    \good{$P=P'$}
    
    \bad{$P\cap P' = \emptyset$ }
    
\end{answers}
\begin{explanations}
On vérifie que $P=P'$.
\end{explanations}

\end{question}



%-------------------------------
\subsection{Plans -- Droites | Difficile | 141.03, 141.04}
 
\begin{question} 
\qtags{motcle=équation droite/plan}

Soit $P$ et $P'$ deux plans non parallèles d'équations :  $ax+by+cz+d =0$ et $a'x+b'y+c'z+d' =0$ respectivement. Soit $D=P\cap P'$ et $Q$ un plan contenant $D$. Quelles sont les assertions vraies ?
\begin{answers}
     
     \bad{Une équation cartésienne de $Q$ est $ax+by+cz+d =0$.}
   
     \bad{Une équation cartésienne de $Q$ est $a'x+b'y+c'z+d' =0$.}  
     
     \good{Une équation cartésienne de $Q$ est  de la forme : $\alpha(ax+by+cz+d)+\beta(a'x+b'y+c'z+d')=0,$  où $ \alpha, \beta \in \Rr$  tels que  $(\alpha a+ \beta a', \alpha b+ \beta b', \alpha c+ \beta c', \alpha d+ \beta d') \neq (0,0,0,0)$.}
 
      \good{Si $Q\neq P'$, une équation cartésienne de $Q$ est de la forme : 
       $(ax+by+cz+d)+\alpha(a'x+b'y+c'z+d')=0, $ où $ \alpha \in \Rr$  tel que $ (a+ \alpha a',  b+ \alpha b', c+ \alpha c',  d+ \alpha d') \neq (0,0,0,0)$.}
    
\end{answers}
\begin{explanations}
$\vec{n}(a,b,c)$ est un vecteur normal à $P$ et $\vec{n'}(a',b',c')$ est un vecteur normal à $P'$, donc un vecteur normal à $Q$ est une combinaison linéaire de $\vec{n}$ et $\vec{n'}$. Par conséquent, une équation cartésienne de $Q$ est de la forme :  $\alpha(ax+by+cz)+\beta(a'x+b'y+c'z) + \gamma =0.$ D'autre part, si $A(x_0,y_0,z_0) \in D$, $A \in Q$. On déduit que $\gamma = \alpha d + \beta d'$.
\end{explanations}

\end{question}
 
 
 
 
 
\begin{question} 
\qtags{motcle=équation droite/plan}

Soit $D$ la droite d'équations :  $ \left\{\begin{array}{ccl}x+z&=&1\\x-y&=&-1 \end{array}\right.$ et $P$ le plan contenant $D$ et perpendiculaire au plan $Q$ d'équation : $x-z+3=0$.  Une équation cartésienne de $P$ est : 
\begin{answers}

    \good{$x+z=1$}
 
      \bad{$x+y=0$}
    
     \bad{$y+z=1$}
   
     \bad{$x-y=-1$}  
   
\end{answers}
\begin{explanations}
$\vec{u}(1,0,-1)$ est un vecteur normal à $ Q$ qui n'appartient pas au  plan vectoriel $x-y=0$. Donc $P$ est différent du plan d'équation : $x-y=-1$ et donc  une équation cartésienne de $P$ est  de la forme : $(x+z-1) + \alpha(x-y+1)=0, \, \alpha \in \Rr$. On calcule $\alpha$ de sorte que $\vec{u}(1,0,-1)$ soit un vecteur de  $P$.
\end{explanations}

\end{question}
 

\begin{question} 
\qtags{motcle=équation droite/plan}

Soit $D$ la droite d'équations :  $ \left\{\begin{array}{ccl}x-y&=&-1\\y-z&=&0 \end{array}\right.$ et $P$ le plan contenant $D$ et parallèle à la droite d'équations  $D' : \left\{\begin{array}{ccl}x+z&=&0\\x-y&=&2 \end{array}\right.$. 
 Une équation cartésienne de $P$ est :
\begin{answers}

       \bad{$x-z=1$}
 
      \bad{$x-y=0$}
    
     \bad{$y-z=0$}
   
     \good{$x-y=-1$}  
   
\end{answers}
\begin{explanations}
$\vec{u}(1,1,-1)$ est un vecteur directeur de la droite $D'$ qui  n'appartient pas au  plan  $y-z=0$ .   Donc $P$ est différent du plan d'équation :  $y-z=0$ et donc une équation cartésienne de $P$ est  de la forme : $(x-y+1) + \alpha(y-z)=0, \, \alpha \in \Rr$. On calcule $\alpha$ de sorte que $\vec{u}(1,1,-1)$ soit un vecteur de  $P$.
\end{explanations}

\end{question}

\begin{question} 
\qtags{motcle=équation droite/plan}

Soit $(P_n), n\in \Nn$, la famille de plans d'équations : $n^2x+(2n-1)y+nz=3$. On note $E$ l'intersection de ces plans, c'est-à-dire $E= \{M(x,y,z) \in \Rr^3; \, M\in P_n, \forall n\in \Nn \}$. Quelles sont les assertions vraies ?
\begin{answers}

       \bad {$E=\emptyset$}
 
      \bad {$E$ est un plan d'équation $x+y+z=3$.}
    
     \bad {$E$ est une droite d'équation $ \left\{\begin{array}{ccl}x+y+z&=&3\\y&=&-3 \end{array}\right.$.}
   
     \good {$E$ est le point de coordonnées $(0,-3,6)$.}  
   
\end{answers}
\begin{explanations}
Soit $M(x,y,z) \in E$, alors   $n^2x+(2n-1)y+nz=3, \, \forall n \in \Nn  \Leftrightarrow  xn^2+ (2y+z)n-y-3=0, \, \forall n \in \Nn  \Leftrightarrow  x=0, 2y+z=0$ et $y+3=0$.
\end{explanations}

\end{question}


\begin{question} 
\qtags{motcle=équation droite/plan}

On considère les droites $D_1 : \left\{\begin{array}{ccl}x&=&z-1\\y&=&2z+1 \end{array}\right.$ et  
$D_2 : \left\{\begin{array}{ccl}y&=&3x\\z&=&1 \end{array}\right.$. Soit $P_1$ et $P_2$ des plans parallèles contenant $D_1$ et $D_2$ respectivement.    Quelles sont les assertions vraies ?
\begin{answers}

       \good{Une équation cartésienne de $P_1$ est $ 3x-y-z+4=0$.}
 
      \bad{Une équation cartésienne de $P_1$ est $ 4x-y-z+5=0$.}
    
     \bad{Une équation cartésienne de $P_2$ est $ 4x-y-z+1=0$.}
   
     \good{Une équation cartésienne de $P_2$ est $ 3x-y-z+1=0$.}  
   
\end{answers}
\begin{explanations}
$D_1$ passe par le point $A_1(-1,1,0)$ et est dirigée par le vecteur $\vec{u_1}(1,2,1)$. 
$D_2$ passe par le point $A_2(0,0,1)$ et est dirigée par le vecteur $\vec{u_2}(1,3,0)$. $P_1$ passe donc par $A_1$ est de vecteur normal  $\vec{n}=\vec{u_1} \wedge \vec{u_2}$ et $P_2$ passe donc par $A_2$ est de vecteur normal  $\vec{n}$.
\end{explanations}

\end{question}


\begin{question} 
\qtags{motcle=équation droite/plan}

Soit  $D_1 : \left\{\begin{array}{ccl}y&=&x+2\\z&=&x \end{array}\right.$,   
$D_2 : \left\{\begin{array}{ccl}y&=&2x+1\\z&=&2x-1 \end{array}\right.$ et $\Delta$ une droite parallèle au plan $(xOy)$ et rencontrant les droites $D_1$, $D_2$ et l'axe $(Oz)$. 
Quelles sont les assertions vraies ?
\begin{answers}

       \bad {Une équation cartésienne de $\Delta$ est : 
     $ \left\{\begin{array}{ccl}y&=&1\\z&=&-1 \end{array}\right.$ ou 
     $ \left\{\begin{array}{ccl}x+y+z&=&0\\z&=&1 \end{array}\right.$.} 
 
      \bad {$\Delta$ est contenu dans le plan $z=-1$ ou $z=1$.}
    
     \good {Une équation cartésienne de $\Delta$ est : 
     $ \left\{\begin{array}{ccl}y&=&0\\z&=&-2 \end{array}\right.$ ou 
     $ \left\{\begin{array}{ccl}y&=&3x\\z&=&1 \end{array}\right.$.}
   
      \good {$\Delta$ est contenu dans le plan $z=-2$ ou $z=1$.}
   
\end{answers}
\begin{explanations}
$\Delta$ est parallèle au plan $(xOy)$ et rencontre l'axe  $(Oz)$, une représentation cartésienne de $\Delta$ est donc  de la forme : 
$ \left\{\begin{array}{ccl}ax+by&=&0\\z&=&c, \quad a,b,c \in \Rr \end{array}\right.$. 

 On peut supposer que $b$ est non nul, sinon, $\Delta$ ne rencontre pas $D_1$ ou $D_2$. Par conséquent, une représentation cartésienne de $\Delta$ est donc  de la forme : 
$ \left\{\begin{array}{ccl}ax+y&=&0\\z&=&b, \quad a,b \in \Rr \end{array}\right.$.

 On calcule $a$ et $b$ pour que $\Delta$ rencontre $D_1$ et $D_2$.
\end{explanations}

\end{question}


%-------------------------------
\subsection{Distance | Facile | 141.05}

\begin{question}
\qtags{motcle=distance}
  
Soit $A(1,1,1)$ et $P$ le plan d'équation cartésienne : $x+y+z+1=0$. La distance de $A$ à $P$ est : 
\begin{answers}

    \bad {$\frac{1}{\sqrt 3}$}

    \bad {$\frac{2}{\sqrt 3}$}
 
   \bad{$\sqrt 3$}
   
    \good{$\frac{4}{\sqrt 3}$}

\end{answers}
\begin{explanations}
Si $P$ est d'équation  $ax+by+cz+d=0$ et $A(x_0,y_0,z_0)$, la distance de $A$ à $P$ est donnée par : $\frac{|ax_0+by_0+cz_0+d|}{\sqrt{a^2+b^2+c^2}}$.
\end{explanations}

\end{question}


\begin{question}
\qtags{motcle=distance}

Soit $A(-1,1,0)$ et $P$ le plan passant par $B(1,0,1)$ et dirigé par les vecteurs $\vec{u}(1,1,1)$ et $\vec{v}(1,0,-1)$. La distance de $A$ à $P$ est : 
\begin{answers}

    \bad {$\frac{1}{\sqrt 6}$}

    \good {$\frac{5}{\sqrt 6}$}
 
     \bad{$\sqrt 6$}
   
    \bad{$\frac{4}{\sqrt 6}$}

\end{answers}
\begin{explanations}
Si $P$ passe par un point $B$ et est dirigé par des vecteurs $\vec{u}$ et $\vec{v}$ et $A$ un point, la distance de $A$ à $P$ est donnée par : $\frac{ \vert \det (\overrightarrow{BA}, \vec{u},\vec{v})\vert}{\Vert \vec{u} \wedge \vec{v}\Vert}$.
\end{explanations}

\end{question}


\begin{question}
\qtags{motcle=distance} 
Soit $A(2,0,1)$ et $D$ la droite d'équations :
$$\left\{\begin{array}{ccl}x+y-z&=&1\\x-y&=&-1 \end{array}\right.$$
  La distance de $A$ à $D$ est : 
\begin{answers}

    \good {$\frac{3}{\sqrt 2}$}

    \bad {$\frac{1}{\sqrt 2}$}
 
     \bad{$\sqrt 3$}
   
    \bad{$\sqrt 2$}
   
\end{answers}
\begin{explanations}
Si $D$ est une droite qui passe par un point $B$ et dirigée par un vecteur $\vec{u}$  et $A$ un point, la distance de $A$ à $D$ est donnée par : $\frac{ \Vert \overrightarrow{BA} \wedge \vec{u} \Vert}{\Vert \vec{u}  \Vert}$.
\end{explanations}

\end{question}

%-------------------------------
\subsection{Distance | Moyen | 141.05}


\begin{question} 
\qtags{motcle=distance}   
     
On considère les droites $D_1 : \left\{\begin{array}{ccl}x&=&1+t\\y&=&-t\\ z&=&1+t,\quad (t \in \Rr)  \end{array}\right.$ et 
$D_2 : \left\{\begin{array}{ccl}y&=&2\\x-z&=&2 \end{array}\right.$
La distance entre $D_1$ et $D_2$ est :
\begin{answers}

    \bad {$0$}

    \bad {$\frac{1}{\sqrt 2}$}
 
   \good{$\sqrt 2$}
    
    \bad{Les droites se rapprochent autant que l'on veut sans se toucher.}
  
\end{answers}
\begin{explanations}
Si $D_1$ passe par un point $A_1$ et est dirigée par un vecteur $\vec{u_1}$ et 
 $D_2$ passe par un point $A_2$ et est dirigée par un vecteur $\vec{u_2}$, la distance entre $D_1$ et $D_2$ est donnée par : $\frac{|\det(\overrightarrow{A_1A_2},\vec{u_1}, \vec{u_2})|}{\Vert \vec{u_1} \wedge \vec{u_2} \Vert }$.
\end{explanations}

\end{question}


\begin{question}
\qtags{motcle=distance}

Soit $D$ la droite passant par le point $A(1,-1,0)$ et dirigée par le vecteur $\vec{u}(1,1,-1)$. Soit $M(1,-1,3)$ un point et $H$ le projeté orthogonal de $M$ sur $D$. Les coordonnées de $H$ sont : 
\begin{answers}

    \bad {$H(0,1,1)$}

    \bad {$H(1,2,1)$}
 
   \good {$H(0,-2,1)$}
    
   \bad {$H(1,-2,1)$}
    
\end{answers}
\begin{explanations}
$H\in D$, donc il existe $t \in \Rr$ tel que $H(1+t,-1+t,-t)$. On calcule $t$ en utilisant l'égalité : $\overrightarrow{HM} \cdot \vec{u} = 0$.
\end{explanations}

\end{question}


\begin{question} 
\qtags{motcle=équation droite/plan}

On considère les droites $D_1 : \left\{\begin{array}{ccl}x+y-z&=&1\\x-y&=&-1 \end{array}\right.$,  
$D_2 : \left\{\begin{array}{ccl}x-y+z&=&-1\\x-z&=&1 \end{array}\right.$
 et $\Delta$ la perpendiculaire commune à $D_1$ et $D_2$. Quelles sont les assertions vraies ?
\begin{answers}

    \bad {Une représentation cartésienne de $\Delta$ est :
 $$\left\{\begin{array}{ccl}x+5y-4z-5&=&0\\x-4y+5z+5&=&0 \end{array}\right.$$}

    \good {Une représentation cartésienne de $\Delta$ est :
 $$\left\{\begin{array}{ccl}x+7y-4z-7&=&0\\x-4y+7z+7&=&0 \end{array}\right.$$}


    \bad { $\Delta$ est contenu dans le plan d'équation $ x+5y-4z-5=0$.}

 
   \good { $\Delta$ est contenu dans le plan d'équation $x-4y+7z+7=0$.}
    
\end{answers}
\begin{explanations}
Soit $D_1$ une droite passant par un point $A_1$ et dirigée par un vecteur $\vec{u_1}$ et 
 $D_2$ une droite passant par un point $A_2$ et dirigée par un vecteur $\vec{u_2}$, telles que $D_1$ et $D_2$ ne soient pas parallèles. Soit $P_1$ le plan passant par $A_1$ et dirigé par les vecteurs $\vec{u_1}$ et $\vec{u_1} \wedge \vec{u_2}$ et 
 $P_2$ le plan passant par $A_2$ et dirigé par les vecteurs $\vec{u_2}$ et $\vec{u_1} \wedge \vec{u_2}$. Alors, la perpendiculaire commune à $D_1$ et $D_2$ est l'intersection de $P_1$ et $P_2$.
\end{explanations}

\end{question}


%-------------------------------
\subsection{Distance | Difficile | 141.05}


\begin{question} 
\qtags{motcle=distance}

Soit  $A(1,1,1)$ un point,  $D$ la droite  $ : \left\{\begin{array}{ccl}x&=&1+z\\y&=&z \end{array}\right.$ et $P$ un plan contenant $D$ et tel que la distance de $A$ à $P$ soit égale à   $\frac{1}{\sqrt 2}$. Une équation cartésienne de $P$ est :
\begin{answers}

       \bad {$x+z+1=0$ ou $x+y+1=0$}
     
 
      \bad {$x-z+1=0$ ou $x-y=0$}
    
     \bad {$z=1$ ou $x=1$}
   
      \good {$x-z=1$ ou $x-y=1$}
   
\end{answers}
\begin{explanations}
$P$ est différent du plan d'équation  $y-z=0$ et $D \subset P$, une équation cartésienne de $P$ est donc de la forme : $(x-z-1)+ \alpha (y-z)=0$. On calcule $\alpha$ pour que la distance de $A$ à $P$ soit égal à $\frac{1}{\sqrt 2}$.
\end{explanations}

\end{question}


\begin{question} 
\qtags{motcle=distance}

Soit  $P_1 : z+3=0$ et $P_2 : 2x+y+2z-1=0$ des plans et $\pi$  un plan bissecteur de $P_1$ et $P_2$, c'est-à-dire :   $M \in \pi$  si et seulement si $M$ est  à la même distance de $P_1$ et de  $P_2$. Une équation cartésienne de $\pi$ est : 
\begin{answers}

       \good {$2x+y-z-10=0$ ou $2x+y+5z+8=0$}
     
       \bad {$x+y-z-1=0$ ou $x+y+z+1=0$}
       
       \bad {$2x+y+z+8=0$ ou $2x-y+5z+7=0$}
       
       \bad {$x+y-z-4=0$ ou $x+y+3z-8=0$}
         
\end{answers}
\begin{explanations}
$M(x,y,z) \in \pi \Leftrightarrow |z+3|=\frac{|2x+y+2z-1|}{3}$.
\end{explanations}

\end{question}


\begin{question} 
\qtags{motcle=distance}

Soit  $E$ l'ensemble des points situés à la même distance  des axes de coordonnées. Quelles sont les assertions vraies ?
\begin{answers}

       \bad{$E$ est une droite.}
     
       \good{$E$ est une réunion de droites.}
       
       \bad{$M(x,y,z) \in E \Leftrightarrow x=y=z$}
       
       \good{$M(x,y,z) \in E \Leftrightarrow |x|=|y|=|z|$}
         
\end{answers}
\begin{explanations}
$M(x,y,z) \in E \Leftrightarrow   \Vert \overrightarrow {OM} \wedge \vec{i} \Vert
 =  \Vert \overrightarrow {OM} \wedge \vec{j} \Vert =  \Vert \overrightarrow {OM} \wedge \vec{k} \Vert$.
\end{explanations}

\end{question}


\begin{question} 
\qtags{motcle=distance}

Soit  $D$ la droite : $\left\{\begin{array}{ccl}x&=&-1+3t\\y&=&1\\z&=&-1-t, \quad t\in \Rr  \end{array}\right.$ et $P$ un plan contenant $D$ à une distance de $1$ de l'origine. Une équation cartésienne de $P$ est : 
\begin{answers}
      \bad{$y=1$} 
 
      \good{$y=1$ ou $4x+3y+12z+13=0$}
    
      \bad {$y=1$ ou $x=1$}
   
      \bad {$x=1$ ou $y=1$ ou $z=1$} 
   
\end{answers}
\begin{explanations}
Une représentation cartésienne de $D$ est : $\left\{\begin{array}{ccl}y&=&1\\x+3z+4&=&0  \end{array}\right.$. $P$ est différent du plan $x+3z+4=0$ et $P$ contient $D$, une équation cartésienne de $P$ est donc de la forme : $(y-1)+\alpha(x+3z+4)=0$. On calcule $\alpha$ de sorte que la distance de $P$ à l'origine soit égale à $1$.
\end{explanations}

\end{question}







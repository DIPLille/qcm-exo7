\qcmtitle{Polynômes}

\qcmauthor{Arnaud Bodin, Abdellah Hanani, Mohamed Mzari}


%%%%%%%%%%%%%%%%%%%%%%%%%%%%%%%%%%%%%%%%%%%%%%%%%%%%%%%%%%%%
\section{Polynômes -- Fractions rationnelles | 105}

\qcmlink[cours]{http://exo7.emath.fr/cours/ch_polynomes.pdf}{Polynômes}

\qcmlink[video]{http://youtu.be/dDKI3jkMjfw}{Définitions}

\qcmlink[video]{http://youtu.be/CnMrf9aW-LU}{Arithmétique des polynômes}

\qcmlink[video]{http://youtu.be/qCMnvqc2t8A}{Racine d'un polynôme, factorisation}

\qcmlink[video]{http://youtu.be/wf-eEQPBX0Y}{Fractions rationnelles}

\qcmlink[exercices]{http://exo7.emath.fr/ficpdf/fic00007.pdf}{Polynômes}

\qcmlink[exercices]{http://exo7.emath.fr/ficpdf/fic00008.pdf}{Fractions rationnelles}


%-------------------------------
\subsection{Polynômes | Facile | 105.05}

\begin{question}
\qtags{motcle=produit}

Soit $P(X) = 2X^5+3X^2+X$ et $Q(X) = 3X^2-2X+3$.
Quelles sont les assertions vraies concernant le polynôme produit $P(X)\times Q(X)$ ?
\begin{answers}
    \bad{Le coefficient dominant est $5$.}

    \good{Le coefficient du monôme $X^3$ est $-3$.}

    \bad{Le coefficient du terme constant est $3$.}

    \good{Le produit est la somme de $7$ monômes ayant un coefficient non nuls.}    
\end{answers}

\begin{explanations}
$P(X)\times Q(X) = 6 X^7 - 4 X^6 + 6 X^5 + 9 X^4 - 3 X^3 + 7 X^2 + 3 X$.
\end{explanations}
\end{question}


\begin{question}
\qtags{motcle=produit}

Soit $P(X) = X^3-3X^2+2$ et $Q(X) = X^3-X+1$.
Quelles sont les assertions vraies ?
\begin{answers}
    \bad{Le polynôme $P(X) \times Q(X)$ est de degré $9$.}

    \bad{Le coefficient du monôme $X^2$ dans le produit $P(X) \times Q(X)$ est $3$.}

    \good{Le polynôme $P(X) + Q(X)$ est de degré $3$.}    
    
    \bad{Le polynôme $P(X) - Q(X)$ est de degré $3$.}
\end{answers}
\begin{explanations}
$P(X)\times Q(X) = X^6 - 3 X^5 - X^4 + 6 X^3 - 3 X^2 - 2 X + 2$, 
$P(X) + Q(X) = 2 X^3 - 3 X^2 - X + 3$,
$P(X) - Q(X) = -3 X^2 + X + 1$.
\end{explanations}
\end{question}


\begin{question}
\qtags{motcle=degré}

Soient $P(X)$ et $Q(X)$ deux polynômes unitaires de degré $n\ge1$.
Quelles sont les assertions vraies ?
\begin{answers}
    \good{$P+Q$ est un polynôme de degré $n$.}

    \bad{$P-Q$ est un polynôme de degré $n$.}
    
    \good{$P \times Q$ est un polynôme de degré $n+n=2n$.}

    \bad{$P/Q$ est un polynôme de degré $n-n=0$.}  
\end{answers}
\begin{explanations}
Le quotient de deux polynômes n'est pas un polynôme.
\end{explanations}
\end{question}


%-------------------------------
\subsection{Polynômes | Moyen | 105.05}


\begin{question}
\qtags{motcle=degré}

Soit $P$ un polynôme de degré $\ge 2$.
Quelles sont les assertions vraies, quel que soit le polynôme $P$ ?
\begin{answers}
    \good{$\deg( P(X) \times (X^2-X+1) ) = \deg P(X) + 2$}

    \bad{$\deg( P(X) + (X^2-X+1) ) = \deg P(X)$}

    \bad{$\deg( P(X)^2 ) = (\deg P(X))^2$}

    \good{$\deg( P(X^2) ) = 2\deg P(X)$}    
\end{answers}
\begin{explanations}
On a la formule $\deg(P\times Q) = \deg P + \deg Q$ mais, il n'y a pas de formule pour la somme, car $\deg(P + Q)$ peut être strictement plus petit que $\deg P$ et $\deg Q$.
\end{explanations}
\end{question}


\begin{question}
\qtags{motcle=dérivée}

Soit $P(X) = \sum_{k=0}^n a_k X^k$. On associe le polynôme dérivé :
$P'(X) = \sum_{k=1}^n ka_k X^{k-1}$.
\begin{answers}
    \bad{Le polynôme dérivé de $P(X) = X^5-2X^2+1$ est $P'(X)=5X^4-2X$.}

    \bad{Le seul polynôme qui vérifie $P'(X)=0$ est $P(X)=1$.}

    \good{Si $P'(X)$ est de degré $7$, alors $P(X)$ est de degré $8$.}

    \bad{Si le coefficient constant de $P$ est nul, alors c'est aussi le cas pour $P'$.}   
\end{answers}
\begin{explanations}
Le polynôme dérivé s'obtient comme si on dérivait la fonction $X \mapsto P(X)$. 
\end{explanations}
\end{question}


%-------------------------------
\subsection{Polynômes | Difficile | 105.05}


\begin{question}
\qtags{motcle=coefficient}

Soit $P(X) = X^n + a_{n-1}X^{n-1} + \cdots + a_1X+a_0$ un polynôme de $\Rr[X]$ de degré $n \ge 1$. À ce polynôme $P$ on associe un nouveau polynôme $Q$, défini par $Q(X) = P(X - \frac{a_{n-1}}{n})$.

Quelles sont les assertions vraies ?
\begin{answers}
    \bad{Si $P(X) = X^2+3X+1$ alors $Q(X) = X^2-2X$.}

    \good{Si $P(X) = X^3-3X^2+2$ alors $Q(X) = X^3-3X$.}

    \bad{Le coefficient constant du polynôme $Q$ est toujours nul.}

    \good{Le coefficient du monôme $X^{n-1}$ de $Q$ est toujours nul.}    
\end{answers}
\begin{explanations}
Cette transformation est faite afin que le coefficient du monôme $X^{n-1}$ de $Q$ soit toujours nul.
\end{explanations}
\end{question}


\begin{question}
\qtags{motcle=dérivée}

Soit $P(X) = \sum_{k=0}^n a_k X^k$. On associe le polynôme dérivé :
$P'(X) = \sum_{k=1}^n ka_k X^{k-1}$. Quelles sont les assertions vraies ?
\begin{answers}
    \good{Si $P$ est de degré $n\ge1$ alors $P'$ est de degré $n-1$.}

    \bad{Si $P'(X) = nX^{n-1}$ alors $P(X) = X^n$.}

    \good{Si $P'=P$ alors $P=0$.}

    \bad{Si $P'-Q'=0$ alors $P-Q=0$.}   
\end{answers}
\begin{explanations}
C'est comme pour les primitives, il ne faut pas oublier la constante :
Si $P'=Q'$ alors $P=Q +c$.
\end{explanations}
\end{question}



\begin{question}
\qtags{motcle=produit}

Soit $A(X) = \sum_{i=0}^n a_i X^i$.
Soit $B(X) = \sum_{j=0}^m b_j X^j$.
Soit $C(X) = A(X) \times B(X) = \sum_{k=0}^{m+n} c_k X^k$.
Quelles sont les assertions vraies ?
\begin{answers}
    \bad{$c_k = a_k b_k$}

    \good{$c_k = \sum_{i+j=k} a_ib_j$}

    \bad{$c_k = \sum_{i=0}^k a_ib_i$}
    
    \good{$c_k = \sum_{i=0}^k a_ib_{k-i}$}
\end{answers}
\begin{explanations}
La formule (à connaître) est 
$$c_k = \sum_{i+j=k} a_ib_j = \sum_{i=0}^k a_ib_{k-i}.$$
\end{explanations}
\end{question}


%-------------------------------
\subsection{Arithmétique des polynômes | Facile | 105.01, 105.02}


\begin{question}
\qtags{motcle=division euclidienne}

Soient $A,B$ deux polynômes, avec $B$ non nul. 
Soit $A = B \times Q + R$ la division euclidienne de $A$ par $B$. 
\begin{answers}
    \good{Un tel $Q$ existe toujours.}

    \good{S'il existe, $Q$ est unique.}

    \good{On a toujours $\deg Q \le \deg A$.}

    \bad{On a toujours $\deg Q \le \deg B$.}    
\end{answers}
\begin{explanations}
La division euclidienne $A = B \times Q + R$ existe toujours, $Q$ et $R$ sont uniques et bien sûr $\deg Q \le \deg A$.
\end{explanations}
\end{question}


\begin{question}
\qtags{motcle=division euclidienne}

Soient $A,B$ deux polynômes, avec $B$ non nul. 
Soit $A = B \times Q + R$ la division euclidienne de $A$ par $B$.

\begin{answers}
    \good{Un tel $R$ existe toujours.}

    \good{S'il existe, $R$ est unique.}

    \bad{On a toujours $\deg R < \deg A$ (ou bien $R$ est nul).}

    \good{On a toujours $\deg R < \deg B$ (ou bien $R$ est nul).}    
\end{answers}
\begin{explanations}
La division euclidienne $A = B \times Q + R$ existe toujours, $Q$ et $R$ sont uniques et par définition de la division euclidienne $R$ est nul ou bien 
$\deg R < \deg B$.
\end{explanations}
\end{question}


\begin{question}
\qtags{motcle=division euclidienne}

Soient $A(X) = 2 X^4 + 3 X^3 - 8 X^2 - 2 X + 1$ et $B(X) = X^2+3X+1$. Soit $A = BQ+R$ la division euclidienne de $A$ par $B$.
\begin{answers}
    \bad{Le coefficient du monôme $X^2$ de $Q$ est $1$.}

    \bad{Le coefficient du monôme $X$ de $Q$ est $3$.}

    \bad{Le coefficient du monôme $X$ de $R$ est $2$.}

    \good{Le coefficient constant de $R$ est $2$.}   
\end{answers}
\begin{explanations}
Faire le calcul !
$Q(X) = 2X^2-3X-1$, $R(X) = 4X+2$.
\end{explanations}
\end{question}




%-------------------------------
\subsection{Arithmétique des polynômes | Moyen | 105.01, 105.02}

\begin{question}
\qtags{motcle=division euclidienne}

Soient $A(X) = X^6 - 7 X^5 + 10 X^4 + 5 X^3 - 23 X^2 + 5$ et $B(X) = X^3-5X^2+1$. Soit $A = BQ+R$ la division euclidienne de $A$ par $B$.
\begin{answers}
    \bad{Le coefficient du monôme $X^2$ de $Q$ est $0$.}

    \good{Le coefficient du monôme $X$ de $Q$ est $0$.}

    \bad{Le coefficient du monôme $X$ de $R$ est $-1$.}

    \good{Le coefficient constant de $R$ est $1$.}   
\end{answers}
\begin{explanations}
Faire le calcul !
$Q(X) = X^3-2X^2+4$, $R(X) = -X^2+1$.
\end{explanations}
\end{question}

\begin{question}
\qtags{motcle=pgcd}

Soient $A(X) = X^4 - 2 X^3 - 4 X^2 + 2 X + 3$ et 
$B(X) = X^4 - 2 X^3 - 3 X^2$ des polynômes de $\Rr[X]$.
Notons $D$ le pgcd de $A$ et $B$.
Quelles sont les affirmations vraies  ?
\begin{answers}
    \bad{$X-1$ divise $D$.}

    \good{$X+1$ divise $D$.}

    \good{$D(X) = (X-3)(X+1)$.}

    \bad{$D(X) = (X-3)(X+1)^2$.}  
\end{answers}
\begin{explanations}
$A(X) = (X-3)(X+1)^2(X-1)$, $B(X) = X^2(X-3)(X+1)$,
le pgcd est $D = (X-3)(X+1)$. 
\end{explanations}
\end{question}


\begin{question}
\qtags{motcle=pgcd}

Quelles sont les affirmations vraies pour des polynômes de $\Rr[X]$ ?
\begin{answers}
    \bad{Le pgcd de $(X-1)^2(X-3)^3(X^2+X+1)^3$ et
    $(X-1)^2(X-2)(X-3)(X^2+X+1)^2$ est $(X-1)^2(X-3)(X^2+X+1)$.}
    
    \bad{Le ppcm de $(X-1)^2(X-3)^3(X^2+X+1)^3$ et
    $(X-1)^2(X-2)(X-3)(X^2+X+1)^2$ est $(X-1)^2(X-2)(X-3)^3(X^2+X+1)^2$.}

    \good{Le pgcd de $(X-1)^2(X^2-1)^3$ et
    $(X-1)^4(X+1)^5$ est $(X-1)^4(X+1)^3$.}
    
    \good{Le ppcm de $(X-1)^2(X^2-1)^3$ et
    $(X-1)^4(X+1)^5$ est $(X-1)^5(X+1)^5$.} 
\end{answers}
\begin{explanations}
Le pgcd s'obtient en prenant le minimum entre les exposants, le ppcm en prenant le maximum. Attention $X^2-1=(X-1)(X+1)$.
\end{explanations}
\end{question}




%-------------------------------
\subsection{Arithmétique des polynômes | Difficile | 105.01, 105.02}

\begin{question}
\qtags{motcle=division euclidienne}

Soit $A$ un polynôme de degré $n\ge1$. Soit $B$ un polynôme de degré $m\ge1$, avec $m \le n$.
Soit $A = B \times Q + R$ la division euclidienne de $A$ par $B$. On note
$q = \deg Q$ et $r = \deg R$ (avec $r=-\infty$ si $R=0$).
Quelles sont les assertions vraies (quelque soient $A$ et $B$) ?
\begin{answers}
    \good{$q = n-m$}

    \good{$r < m$}

    \bad{$r=0 \implies A$ divise $B$.}

    \bad{$n = mq + r$}    
\end{answers}
\begin{explanations}
On a $\deg R < \deg B$. Il ne faut pas confondre $R=0$ et $r=0$.
En plus $\deg(A) = \deg(B\times Q) = \deg(A) + \deg(Q)$.
\end{explanations}
\end{question}


\begin{question}
\qtags{motcle=division euclidienne}

Soit $n\ge2$. Soit $A(X) = X^{2n}+X^{2n-2}$. Soit $B(X) = X^{n}+X^{n-1}$. Soit $A = BQ + R$ la division euclidienne de $A$ par $B$. 
\begin{answers}
    \good{Le coefficient de $X^n$ de $Q$ est $1$.}

    \bad{Le coefficient de $X^{n-1}$ de $Q$ est $1$.}    

    \good{Le coefficient de $X^{n-2}$ de $Q$ est $2$.}       

    \good{$R$ est constitué d'un seul monôme.}
   
\end{answers}
\begin{explanations}
$Q(X) = X^n-X^{n-1}+2X^{n-2}-2X^{n-3}+\cdots$. $R(X) = \pm 2 X^{n-1}$.
\end{explanations}
\end{question}



\begin{question}
\qtags{motcle=pgcd}

Soit $A(X) = X^4-X^2$. Soit $B(X) = X^2+X-2$.
Soit $D$ le pgcd de $A$ et $B$ dans $\Rr[X]$.
\begin{answers}
    \bad{$D(X) = 1$}

    \good{Il existe $U,V \in \Rr[X]$ tels que $AU+BV = X-1$.}

    \good{Il existe $u \in \Rr$ et $V \in \Rr[X]$ tels que $Au+BV = X-1$.}

    \bad{Il existe $U\in \Rr[X]$ et $v \in \Rr$ tels que $AU+Bv = X-1$.}
\end{answers}
\begin{explanations}
$A(X)=X^2(X-1)^2$, $B(X)=(X-1)(X+2)$, $D(X)=X-1$. $U(X)= -\frac14$, $V(X)=\frac14(X^2-X+2)$ donnent $AU+BV=D$.
\end{explanations}
\end{question}

%-------------------------------
\subsection{Racines, factorisation | Facile | 105.03}


\begin{question}
\qtags{motcle=racine}

Soit $P \in \Rr[X]$ un polynôme de degré $8$.
Quelles sont les affirmations vraies ?
\begin{answers}
    \bad{$P$ admet exactement $8$ racines réelles (comptées avec multiplicité).}

    \bad{$P$ admet au moins une racine réelle.}

    \good{$P$ admet au plus $8$ racines réelles (comptées avec multiplicité).}

    \bad{$P$ admet au moins $8$ racines réelles (comptées avec multiplicité).}
    
\end{answers}
\begin{explanations}
Il y a au plus $\deg P$ racines réelles (comptées avec multiplicité).
\end{explanations}
\end{question}


\begin{question}
\qtags{motcle=racine}

Soit $P(X) = X^7 - 5 X^5 - 5 X^4 + 4 X^3 + 13 X^2 + 12 X + 4$.
\begin{answers}

    \good{$-1$ est une racine de $P$.}

    \bad{$0$ est une racine de $P$.}
    
    \bad{$1$ est une racine de $P$.}
    
    \good{$2$ est une racine de $P$.} 
\end{answers}
\begin{explanations}
Calculer $P(\alpha)$. En fait $P(X) = (X-2)^2(X+1)^3(X^2+X+1)$.
\end{explanations}
\end{question}



\begin{question}
\qtags{motcle=polynôme irréductible}

Quelles sont les affirmations vraies ?
\begin{answers}
    \bad{$2X^2+3X+1$ est irréductible sur $\Qq$.}

    \good{$2X^2-3X+2$ est irréductible sur $\Rr$.}

    \bad{$2X^2-X+3$ est irréductible sur $\Cc$.}

    \bad{$X^3+X^2+X+4$ est irréductible sur $\Rr$.}  
\end{answers}
\begin{explanations}
Sur $\Cc$ les irréductibles sont de degré $1$. Sur $\Rr$ ils sont de degré 1, ou bien de degré $2$ à discriminant strictement négatif.
\end{explanations}
\end{question}



%-------------------------------
\subsection{Racines, factorisation | Moyen | 105.03}


\begin{question}
\qtags{motcle=racine}

Soit $P \in \Rr[X]$ un polynôme de degré $2n+1$ ($n\in\Nn^*$).
Quelles sont les affirmations vraies ?
\begin{answers}
    \good{$P$ peut admettre une racine complexe, qui ne soit pas réelle.}

    \good{$P$ admet au moins une racines réelle.}

    \bad{$P$ admet au moins deux racines réelles (comptées avec multiplicités).}

    \good{$P$ peut avoir $2n+1$ racines réelles distinctes.}
\end{answers}
\begin{explanations}
Il y a au plus $\deg P$ racines réelles (comptées avec multiplicité). Mais ici le degré est impair, donc $P$ admet au moins une racine réelle.
\end{explanations}
\end{question}


\begin{question}
\qtags{motcle=racine}

Soit $P(X) = X^6 + 4 X^5 + X^4 - 10 X^3 - 4 X^2 + 8 X$.
\begin{answers}
    \bad{$-1$ est une racine double.}

    \bad{$0$ est une racine double.}

    \good{$1$ est une racine double.}

    \bad{$-2$ est une racine double.}   
\end{answers}
\begin{explanations}
Pour une racine double il faut $P(a)=0$, $P'(a)=0$ et $P''(a)\neq0$.
En fait $P(X) = X(X+2)^3(X-1)^2$.
\end{explanations}
\end{question}




%-------------------------------
\subsection{Racines, factorisation | Difficile | 105.03}


\begin{question}
\qtags{motcle=polynôme irréductible}

Soit $P \in \Qq[X]$ un polynôme de degré $n$.
\begin{answers}
    \good{$P$ peut avoir des racines dans $\Rr$, mais pas dans $\Qq$.}

    \good{Si $z\in \Cc\setminus\Rr$ est une racine de $P$, alors $\bar z$ aussi.}

    \bad{Les facteurs irréductibles de $P$ sur $\Qq$ sont de degré $1$ ou $2$.}

    \bad{Les racines réelles de $P$ sont de la forme $\alpha + \beta\sqrt{\gamma}$, $\alpha,\beta,\gamma \in \Qq$.}   
\end{answers}
\begin{explanations}
Sur $\Qq$ les facteurs irréductibles peuvent être de n'importe quel degré.
\end{explanations}
\end{question}



\begin{question}
\qtags{motcle=racine}

Soit $P \in \Kk[X]$ un polynôme de degré $n\ge1$.
Quelles sont les affirmations vraies ?
\begin{answers}
    \good{$a$ racine de $P$ $\iff$ $X-a$ divise $P$.}

    \good{$a$ racine de $P$ de multiplicité $\ge k$ $\iff$ $(X-a)^k$ divise $P$.}
    
    \bad{$a$ racine de $P$ de multiplicité $\ge k$ $\iff$ 
    $P(a) = 0$, $P'(a)=0$, ..., $P^{(k)}(a)=0$.}

    \good{La somme des multiplicités des racines est $\le n$.}
   
\end{answers}
\begin{explanations}
$a$ racine de $P$ de multiplicité $\ge k$ $\iff$ $(X-a)^k$ divise $P$ $\iff$ $P(a) = 0$, $P'(a)=0$, ..., $P^{(k-1)}(a)=0$.
\end{explanations}
\end{question}

%-------------------------------
\subsection{Fractions rationnelles | Facile | 105.04}

\begin{question}
\qtags{motcle=éléments simples}

Quelles sont les affirmations vraies ?
\begin{answers}
    \bad{Les éléments simples sur $\Cc$ sont de la forme $\frac{a}{X-\alpha}$, $a,\alpha \in \Cc$.}

    \bad{Les éléments simples sur $\Cc$ sont de la forme $\frac{a}{(X-\alpha)^k}$, $a,\alpha \in \Rr$, $k\in\Nn^*$.}

    \good{Les éléments simples sur $\Rr$ peuvent être de la forme $\frac{a}{(X-\alpha)^k}$, $a,\alpha \in \Rr$.}

    \bad{Les éléments simples sur $\Rr$ peuvent être de la forme $\frac{aX+b}{X-\alpha}$, $a,b,\alpha \in \Rr$.}
 
\end{answers}
\begin{explanations}
Sur $\Cc$ les éléments simples sont de la forme $\frac{a}{(X-\alpha)^k}$, $a,\alpha \in \Cc$, $k\in\Nn^*$.
Sur $\Rr$ les éléments simples sont de la forme $\frac{a}{(X-\alpha)^k}$, $a,\alpha \in \Rr$, $k \in \Nn^*$ ou bien
$\frac{aX+b}{(X^2+\alpha X+\beta)^k}$, $a,b,\alpha,\beta \in \Rr$, $k \in \Nn^*$ avec 
$X^2+\alpha X+\beta$ sans racines réelles.
\end{explanations}
\end{question}


\begin{question}
\qtags{motcle=éléments simples}

Soient $P(X)=X-1$, $Q(X)=(X+1)^2(X^2+X+1)$. On décompose la fraction $F = \frac{P}{Q}$ sur $\Rr$.
\begin{answers}
    \good{La partie polynomiale est nulle.}
    
    \bad{Il peut y avoir un élément simple $\frac{a}{X-1}$.} 
    
    \bad{Il peut y avoir un élément simple $\frac{a}{X+1}$ mais pas  $\frac{a}{(X+1)^2}$.}
    
    \good{Il peut y avoir un élément simple $\frac{aX+b}{X^2+X+1}$ mais pas  $\frac{aX+b}{(X^2+X+1)^2}$.}
 
\end{answers}
\begin{explanations}
$\frac{P(X)}{Q(X)} = \frac{X-1}{(X+1)^2(X^2+X+1)}
= \frac{-1}{X+1}+\frac{-2}{(X+1)^2}+\frac{X+2}{X^2+X+1}$.
\end{explanations}
\end{question}



%-------------------------------
\subsection{Fractions rationnelles | Moyen | 105.04}


\begin{question}
\qtags{motcle=division euclidienne}

Soit $\frac{P(X)}{Q(X)}$ une fraction rationnelle. On note $E(X)$ sa partie polynomiale (appelée aussi partie entière).
\begin{answers}
    \good{Si $\deg P < \deg Q$ alors $E(X) = 0$.}

    \good{Si $\deg P \ge \deg Q$ alors $\deg E(X) = \deg P - \deg Q$.}

    \bad{Si $P(X) = X^3+X+2$ et $Q(X) = X^2-1$ alors $E(X) = X+1$.}

    \good{Si $P(X) = X^5+X-2$ et $Q(X) = X^2-1$ alors $E(X) = X^3+X$.}   
\end{answers}
\begin{explanations}
La partie entière s'obtient comme le quotient de la division euclidienne de $P$ par $Q$.
\end{explanations}
\end{question}


\begin{question}
\qtags{motcle=éléments simples}

Soit $P(X)=3X$ et $Q(X) = (X-2)(X-1)^2(X^2-X+1)$.
On écrit 
$$\frac{P(X)}{Q(X)} = \frac{a}{X-2} + \frac{b}{X-1} +  \frac{c}{(X-1)^2}
+ \frac{dX+e}{X^2-X+1}.$$
Quelles sont les affirmations vraies ? 
\begin{answers}
    \bad{En multipliant par $X-2$, puis en évaluant en $X=2$, j'obtiens $a=1$.}

    \good{En multipliant par $(X-1)^2$, puis en évaluant en $X=1$, j'obtiens $c=-3$.}

    \good{En multipliant par $X$, puis en faisant tendre $X \to +\infty$, j'obtiens la relation $a+b+d=0$.}

    \bad{En évaluant en $X=0$, j'obtiens la relation $a+b+c+e=0$.}    
\end{answers}
\begin{explanations}
$\frac{P(X)}{Q(X)} = \frac{3X}{(X-2)(X-1)^2(X^2-X+1)}
=\frac{2}{X-2} + \frac{-3}{X-1} +  \frac{-3}{(X-1)^2}
+ \frac{X+1}{X^2-X+1}$.
\end{explanations}
\end{question}



%-------------------------------
\subsection{Fractions rationnelles | Difficile | 105.04}



\begin{question}
\qtags{motcle=éléments simples}

Soit $F(X) = \dfrac{1}{(X^2+1)X^3}$.
On écrit 
$$F(X) = \frac{a}{X} + \frac{b}{X^2} +  \frac{c}{X^3}
+ \frac{dX+e}{X^2+1}.$$
Quelles sont les affirmations vraies ?
\begin{answers}
    \good{$c=1$}

    \bad{$b=1$}

    \bad{$a=1$}

    \good{$e=0$}   
\end{answers}
\begin{explanations}
On profite que $F$ est impaire pour déduire $b=0$, $e=0$.
$F(X) = \dfrac{1}{(X^2+1)X^3} = \frac{-1}{X}  +  \frac{1}{X^3}
+ \frac{X}{X^2+1}.$
\end{explanations}
\end{question}


\begin{question}
\qtags{motcle=éléments simples}

Soit $F(X) = \dfrac{X-1}{X(X^2+1)^2}$.
On écrit 
$$F(X) = \frac{a}{X} + \frac{bX+c}{X^2+1} +  \frac{dX+e}{(X^2+1)^2}.$$
Quelles sont les affirmations vraies ?
\begin{answers}
    \good{$a=-1$}

    \bad{$d=0$ et $e=0$}

    \bad{$b=0$ et $c=0$}

    \bad{$b=0$ et $d=0$}
    
\end{answers}
\begin{explanations}
$F(X) = \dfrac{1}{X(X^2+1)^2} = \frac{-1}{X} + \frac{X}{X^2+1} +  \frac{X+1}{(X^2+1)^2}.$
\end{explanations}
\end{question}


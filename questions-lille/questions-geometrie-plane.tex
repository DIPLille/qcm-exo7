\qcmtitle{Géométrie du plan}

\qcmauthor{Arnaud Bodin, Abdellah Hanani, Mohamed Mzari}



\section{Géométrie du plan | 140}

\qcmlink[exercices]{http://exo7.emath.fr/ficpdf/fic00159.pdf}{Droites du plan ; droites et plans de l'espace}

\subsection{Géométrie du plan | Facile | 140.01, 140.02}

\begin{question}
On considère les points $A(3,0)$ et $B(0,4)$. Quelle est la distance $d$ entre $A$ et $B$ ?
\begin{answers}  
    \bad{$d=3$}
    \bad{$d=4$}
    \good{$d=5$}
    \bad{$d=3+4=7$}
\end{answers}
\begin{explanations}
D'abord, $\overrightarrow{AB}=(-3,4)$. Donc $d=\sqrt{(-3)^2+4^2}=\sqrt{25}=5$.
\end{explanations}
\end{question}

\begin{question}
On considère les vecteurs $\vec{u}=(2,-1)$ et $\vec{v}=(1,-4)$. Quelles sont les bonnes réponses ?
\begin{answers}  
    \bad{La norme de $\vec{u}$ est $\|\vec{u}\|=2-1=1$.}
    \good{La norme de $\vec{u}$ est $\|\vec{u}\|=\sqrt{5}$.}
    \bad{Le produit scalaire de $\vec{u}$ et $\vec{v}$ est $\vec{u}\cdot \vec{v}=(2-1)+(1-4)=-3$.}
    \good{Le produit scalaire de $\vec{u}$ et $\vec{v}$ est $\vec{u}\cdot \vec{v}=6$.}
\end{answers}
\begin{explanations}
Penser aux définitions : si $\vec{u}=(x,y)$ et $\vec{v}=(x',y')$ alors
$\vec{u}\cdot \vec{v} = xx'+yy'$ et $\|\vec{u}\| = \sqrt{\vec{u}\cdot \vec{u}}
= \sqrt{x^2+y^2}$.
\end{explanations}
\end{question}


\begin{question}
On considère les points $A(1,1)$, $B(-1,1)$ et $C(1,-1)$. Quelles sont les bonnes réponses ?
\begin{answers}  
    \bad{Les vecteurs $\overrightarrow{AB}$ et $\overrightarrow{AC}$ sont égaux.}
    \bad{$\overrightarrow{AB}=-\overrightarrow{AC}$}
    \bad{Les vecteurs $\overrightarrow{AB}$ et $\overrightarrow{AC}$ sont colinéaires.}
    \good{Les vecteurs $\overrightarrow{AB}$ et $\overrightarrow{AC}$ sont orthogonaux.}
\end{answers}
\begin{explanations}
D'abord, $\overrightarrow{AB}=(-2,0)$ et $\overrightarrow{AC}=(0,-2)$, et puis le produit scalaire $\overrightarrow{AB}\cdot\overrightarrow{AC}$ est nul. Donc $\overrightarrow{AB}$ et $\overrightarrow{AC}$ sont orthogonaux.
\end{explanations}
\end{question}


\begin{question}
Dans un repère orthonormé direct, on considère le point $A$ de coordonnées polaires $r=2$ et $\displaystyle \theta =\frac{\pi}{6}$. Quelles sont les coordonnées cartésiennes $(x,y)$ de $A$ ?
\begin{answers}  
    \bad{$x=2$ et $y=2$}
    \good{$x=\sqrt{3}$ et $y=1$}
    \bad{$x=1$ et $y=\sqrt{3}$}
    \bad{$x=1$ et $y=1$}
\end{answers}
\begin{explanations}
Les deux systèmes de coordonnées sont reliés par les relations $x=r\cos \theta$ et $y=r\sin \theta $.
\end{explanations}
\end{question}


\begin{question}
Dans un repère orthonormé direct, on considère le point $A(1,1)$. Quelles sont les coordonnées polaires $(r,\theta)$ de $A$ ?
\begin{answers}  
    \bad{$r=1$ et $\theta =1$}
    \bad{$r=2$ et $\theta =0$}
    \good{$r=\sqrt{2}$ et $\displaystyle \theta =\frac{\pi}{4}+2k\pi$, $k\in \Zz$}
    \bad{$r=\sqrt{2}$ et $\theta =0+2k\pi$, $k\in \Zz$}
\end{answers}
\begin{explanations}
D'abord, $r=\sqrt{1^2+1^2}=\sqrt{2}$ et $\theta $ est solution du système : 
$$\left\{\begin{array}{l}\displaystyle \cos \theta =\frac{1}{\sqrt{2}}\\ \\ \displaystyle \sin \theta =\frac{1}{\sqrt{2}}.\end{array}\right.$$
\end{explanations}
\end{question}


\begin{question}
On considère les points $A(0,1)$, $B(2,3)$ et $C(1,1)$. Quelles sont les bonnes réponses ?
\begin{answers}  
    \bad{Les droites $(AB)$ et $(OC)$ sont confondues.}
    \bad{Les droites $(AB)$ et $(OC)$ sont perpendiculaires.}
    \good{Les droites $(AB)$ et $(OC)$ sont parallèles.}
    \bad{Les droites $(AB)$ et $(OC)$ sont sécantes.}
\end{answers}
\begin{explanations}
On a $\overrightarrow{AB}=(2,2)=2\overrightarrow{OC}$. Les droites $(AB)$ et $(OC)$ sont parallèles.
\end{explanations}
\end{question}


\begin{question}
On considère les points $A(-1,-1)$, $B(-1,1)$, $C(1,2)$ et $D(1,0)$. Quelles sont les bonnes réponses ?
\begin{answers}  
    \bad{Les droites $(AB)$ et $(CD)$ sont sécantes.}
    \bad{Les droites $(AB)$ et $(CD)$ sont perpendiculaires.}
    \good{Les droites $(AB)$ et $(CD)$ sont parallèles.}
    \good{$(ABCD)$ est un parallélogramme.}
\end{answers}
\begin{explanations}
On a $\overrightarrow{AB}=(0,2)=-\overrightarrow{CD}$, donc les droites $(AB)$ et $(CD)$ sont parallèles. De plus, $AB=CD$, donc $(ABCD)$ est un parallélogramme.
\end{explanations}
\end{question}


\begin{question}
Soit $D$ la droite passant par l'origine et par le point $A(1,1)$. Quelles sont les bonnes réponses ?
\begin{answers}  
    \good{$\vec{u}(1,1)$ est un vecteur directeur de $D$.}
    \bad{$\vec{u}(1,1)$ est un vecteur normal à $D$.}
    \good{$y=x$ est une équation cartésienne de $D$.}
    \bad{$x+y=0$ est une équation cartésienne de $D$.}
\end{answers}
\begin{explanations}
La droite $D$ est dirigée par le vecteur $\overrightarrow{OA}=\vec{u}(1,1)$ et $M(x,y)\in D \Leftrightarrow \mbox{det}(\overrightarrow{OM},\overrightarrow{OA})=0\Leftrightarrow x-y=0$. Ceci donne une équation cartésienne de $D$.
\end{explanations}
\end{question}


\begin{question}
Soit $D$ la droite passant par les points $A(1,-1)$ et $B(1,1)$. Quelles sont les bonnes réponses ?
\begin{answers}  
    \good{$\vec{u}(0,1)$ est un vecteur directeur de $D$.}
    \bad{$\vec{u}(0,1)$ est un vecteur normal à $D$.}
    \bad{Le point $C(1,0)$ n'appartient pas à $D$.}
    \good{Le point $C(1,0)$ appartient à $D$.}
\end{answers}
\begin{explanations}
Le vecteur $\overrightarrow{AB}=(0,2)$ est un vecteur directeur de $D$. Par ailleurs, $\overrightarrow{AC}=(0,1)=\frac{1}{2}\overrightarrow{AB}$. Donc $C\in D$.
\end{explanations}
\end{question}


\begin{question}
Soit $D$ la droite passant par les points $A(1,-1)$ et $B(1,0)$. Quelles sont les bonnes réponses ?
\begin{answers}  
    \bad{Une équation cartésienne de $D$ est : $x-y+1=0$.}
    \good{Une équation cartésienne de $D$ est : $x-1=0$.}
    \good{$\vec{u}(1,0)$ est un vecteur normal à $D$.}
    \bad{$\vec{u}(1,0)$ est un vecteur directeur de $D$.}
\end{answers}
\begin{explanations}
Les coordonnées de $A$ et $B$ vérifient l'équation $x-1=0$, celle-ci est donc une équation cartésienne de $D$ et $\vec{u}(1,0)$ est un vecteur normal à $D$.
\end{explanations}
\end{question}


\subsection{Géométrie du plan | Moyen | 140.01, 140.02}


\begin{question}
Dans le plan muni d'un repère orthonormé direct $(O,\vec{i},\vec{j})$, on considère les vecteurs $\displaystyle \vec{u}=\left(1,1\right)$ et $\displaystyle \vec{v}=\left(1,\sqrt{3}\right)$. Quel est la mesure $\alpha \in [0,2\pi[$ de l'angle orienté entre $\vec{u}$ et $\vec{v}$ ?
\begin{answers}  
    \bad{$\displaystyle \alpha =\frac{\pi}{4}$}
    \bad{$\displaystyle \alpha =\frac{\pi}{3}$}
    \good{$\displaystyle \alpha =\frac{\pi}{12}$}
    \bad{$\displaystyle \alpha =\frac{7\pi}{12}$}
\end{answers}
\begin{explanations}
Une mesure de l'angle orienté entre $\vec{i}$ et $\vec{u}$ est $\displaystyle a=\frac{\pi}{4}$ et une mesure de l'angle orienté entre $\vec{i}$ et $\vec{v}$ est $\displaystyle b=\frac{\pi}{3}$. Donc $\displaystyle \alpha =b-a=\frac{\pi}{12}$.
\end{explanations}
\end{question}


\begin{question}
Dans le plan muni d'une base orthonormée $(\vec{i},\vec{j})$, on considère les vecteurs $\displaystyle \vec{u}=\left(\frac{1}{\sqrt{2}},a\right)$ et $\displaystyle \vec{v}=\left(a,-\frac{1}{\sqrt{2}}\right)$. Comment choisir le réel $a$ pour que $(\vec{u},\vec{v})$ soit une base orthonormée ?
\begin{answers}  
    \good{$\displaystyle a=\frac{1}{\sqrt{2}}$}
    \good{$\displaystyle a=-\frac{1}{\sqrt{2}}$}
    \bad{$\displaystyle a=\sqrt{2}$}
    \bad{$\displaystyle a=-\sqrt{2}$}
\end{answers}
\begin{explanations}
Pour tout $a\in \Rr$, les vecteurs $\vec{u}$ et $\vec{v}$ sont orthogonaux. Ensuite, $\|\vec{u}\|=\|\vec{v}\|=1$ implique $\displaystyle a=\frac{\pm 1}{\sqrt{2}}$.
\end{explanations}
\end{question}



\begin{question}
Dans le plan muni d'une base orthonormée $(\vec{i},\vec{j})$, on considère les vecteurs $\displaystyle \vec{u}=\left(\frac{1}{2},a\right)$ et $\displaystyle \vec{v}=\left(-\frac{\sqrt{3}}{2},b\right)$. Comment choisir les réels $a$ et $b$ pour que $(\vec{u},\vec{v})$ soit une base orthonormée ?
\begin{answers}  
    \good{$\displaystyle a=\frac{\sqrt{3}}{2}$ et $\displaystyle b=\frac{1}{2}$}
    \bad{$\displaystyle a=\frac{\sqrt{3}}{2}$ et $\displaystyle b=-\frac{1}{2}$}
    \bad{$\displaystyle a=-\frac{\sqrt{3}}{2}$ et $\displaystyle b=\frac{1}{2}$}
    \good{$\displaystyle a=-\frac{\sqrt{3}}{2}$ et $\displaystyle b=-\frac{1}{2}$}
\end{answers}
\begin{explanations}
D'abord, $\displaystyle \|\vec{u}\|=1\Leftrightarrow a=\frac{\pm \sqrt{3}}{2}$, $\displaystyle \|\vec{v}\|=1\Leftrightarrow b=\frac{\pm 1}{2}$ et $\vec{u}\cdot\vec{v}=0$ si, et seulement si, $a$ et $b$ sont de même signe.
\end{explanations}
\end{question}


\begin{question}
Dans le plan muni d'une base orthonormée $(\vec{i},\vec{j})$, on considère deux vecteurs $\displaystyle \vec{u}$ et $\displaystyle \vec{v}$. On suppose que $\|\vec{u}\|=3$, $\|\vec{v}\|=3$ et que l'angle entre ces deux vecteurs est $\displaystyle \frac{\pi}{3}$. Quelle est la norme de $\vec{u}+\vec{v}$ ?
\begin{answers}  
    \bad{$\displaystyle \|\vec{u}+\vec{v}\|=6$}
    \bad{$\displaystyle \|\vec{u}+\vec{v}\|=3$}
    \good{$\displaystyle \|\vec{u}+\vec{v}\|=3\sqrt{3}$}
    \bad{$\displaystyle \|\vec{u}+\vec{v}\|=9$}
\end{answers}
\begin{explanations}
La bilinéarité et la symétrie du produit scalaire donnent
$$\|\vec{u}+\vec{v}\|^2=(\vec{u}+\vec{v})\cdot(\vec{u}+\vec{v})=\|\vec{u}\|^2+\|\vec{v}\|^2+2\vec{u}\cdot\vec{v}.$$
Et puis $\displaystyle \vec{u}\cdot \vec{v}=\|\vec{u}\|\|\vec{v}\|\cos \left(\frac{\pi}{3}\right)=\frac{9}{2}$. Donc $\|\vec{u}+\vec{v}\|^2=9+9+9$.
\end{explanations}
\end{question}


\begin{question}
On considère les points $A(1,1)$, $B(-1,1)$ et $C(1,-1)$. Quelles sont les bonnes réponses ?
\begin{answers}  
    \bad{Les points $A$, $B$ et $C$ sont alignés.}
    \good{$ABC$ est un triangle rectangle en $A$.}
    \bad{$ABC$ est un triangle équilatéral.}
    \good{$ABC$ est un triangle isocèle en $A$.}
\end{answers}
\begin{explanations}
On a $\overrightarrow{AB}=(-2,0)$ et $\overrightarrow{AC}=(0,-2)$. Les points $A$, $B$ et $C$ ne sont pas alignés. De plus, $\|\overrightarrow{AB}\|=2=\|\overrightarrow{AC}\|$ donc $ABC$ est isocèle en $A$ et $\overrightarrow{AB}.\overrightarrow{AC}=0$, donc $ABC$ est rectangle en $A$.
\end{explanations}
\end{question}


\begin{question}
Soit $D$ la droite définie par le paramétrage :
$$\left\{\begin{array}{ccl}x&=&1+t\\ y&=&2-t,\quad t\in \Rr.
\end{array}\right.$$
Quelles sont les bonnes réponses ?
\begin{answers}  
    \bad{Le point $A(1,1)$ appartient à $D$.}
    \bad{$\vec{u}=(1,-1)$ est un vecteur normal à $D$.}
    \good{Une équation cartésienne de $D$ est : $x+y-3=0$.}
    \bad{$\vec{u}(1,1)$ est un vecteur directeur de $D$.}
\end{answers}
\begin{explanations}
Le vecteur $\vec{u}=(1,-1)$ est un vecteur directeur de $D$. On élimine $t$ en additionnant les deux équations. Ceci donne $x+y=3$ qui est une équation cartésienne de $D$.
\end{explanations}
\end{question}



\begin{question}
Dans le plan rapporté à un repère orthonormé, on considère la droite $D$ passant par les points $A(1,1)$ et $B(2,3)$. Quelles sont les bonnes réponses ?
\begin{answers}  
    \bad{$\vec{u}=(1,2)$ est un vecteur normal à $D$.}
    \good{Une équation cartésienne de $D$ est : $2x-y-1=0$.}
    \bad{Le point $C(1,2)$ appartient à $D$.}
    \good{La distance du point $N(-1,2)$ à la droite $D$ est $\sqrt{5}$.}
\end{answers}
\begin{explanations}
Le vecteur $\overrightarrow{AB}=(1,2)$ dirige $D$ et $M(x,y)\in D\Leftrightarrow \mbox{det} \left(\overrightarrow{AM},\overrightarrow{AB}\right)=0$, c'est-à-dire $2x-y-1=0$. La distance de $N$ à $D$ est donnée par $\displaystyle \frac{|2\times (-1)-2-1|}{\sqrt{2^2+1^2}}=\sqrt{5}$.
\end{explanations}
\end{question}






\begin{question}
Dans le plan rapporté à un repère orthonormé, on considère les points $\displaystyle A(1,2)$, $B(2,1)$ et $C(-2,1)$. Quelle est la distance $d$ du point $C$ à la droite $(AB)$ ?
\begin{answers}  
    \bad{$d=\sqrt{2}$}
    \bad{$d=3$}
    \good{$d=2\sqrt{2}$}
    \bad{$d=\sqrt{10}$}
\end{answers}
\begin{explanations}
Utiliser la formule $\displaystyle d=\frac{\left|\mbox{det}\left(\overrightarrow{AC},\overrightarrow{AB}\right)\right|}{\|\overrightarrow{AB}\|}=2\sqrt{2}$.
\end{explanations}
\end{question}


\begin{question}
Dans le plan rapporté à un repère orthonormé, on considère la droite $D$ d\'efinie par le paramétrage :
$$\left\{\begin{array}{ccl}x&=&1+t\\ y&=&2-t,\quad t\in \Rr.
\end{array}\right.$$
Quelle est la distance $d$ du point $M(2,3)$ à la droite $D$ ?
\begin{answers}  
    \good{$d=\sqrt{2}$}
    \bad{$d=\sqrt{3}$}
    \bad{$d=1$}
    \bad{$d=2$}
\end{answers}
\begin{explanations}
Le point $A(1,2)\in D$ et le vecteur $\vec{v}=(1,-1)$ dirige $D$. On utilise la formule $\displaystyle d=\frac{\left|\mbox{det}\left(\overrightarrow{AM},\vec{v}\right)\right|}{\|\vec{v}\|}=\sqrt{2}$.
\end{explanations}
\end{question}



\begin{question}
Dans le plan muni d'un repère orthonormé $(O,\vec{i},\vec{j})$, on considère les points $\displaystyle A(a,b)$ et $\displaystyle B(1,1)$. Comment choisir les réels $a$ et $b$ pour que l'aire du triangle de sommets $O,A,B$ soit égale à $1$ ?
\begin{answers}  
    \good{$\displaystyle a=2$ et $\displaystyle b=0$}
    \good{$\displaystyle a=2+b$ et $\displaystyle b\in \Rr$}
    \bad{$\displaystyle a=1$ et $\displaystyle b=0$}
    \bad{$\displaystyle a=0$ et $\displaystyle b=1$}
\end{answers}
\begin{explanations}
On doit avoir $\displaystyle 2\mbox{Aire}(OAB)=\left|\mbox{det}\left(\overrightarrow{OA},\overrightarrow{AB}\right)\right|=2$. Ceci donne ($a=2+b$ et $b\in \Rr$).
\end{explanations}
\end{question}



\subsection{Géométrie du plan | Difficile | 140.01, 140.02}


\begin{question}
Dans le plan muni d'un repère orthonormé direct $(O,\vec{i},\vec{j})$, on considère les vecteurs $\displaystyle \vec{u}=\left(\frac{1}{2},a\right)$ et $\displaystyle \vec{v}=\left(-\frac{\sqrt{3}}{2},b\right)$. Comment choisir les réels $a$ et $b$ pour que $(\vec{u},\vec{v})$ soit une base orthonormée directe ?
\begin{answers}  
    \good{$\displaystyle a=\frac{\sqrt{3}}{2}$ et $\displaystyle b=\frac{1}{2}$}
    \bad{$\displaystyle a=\frac{\sqrt{3}}{2}$ et $\displaystyle b=-\frac{1}{2}$}
    \bad{$\displaystyle a=-\frac{\sqrt{3}}{2}$ et $\displaystyle b=\frac{1}{2}$}
    \bad{$\displaystyle a=-\frac{\sqrt{3}}{2}$ et $\displaystyle b=-\frac{1}{2}$}
\end{answers}
\begin{explanations}
D'abord, $\displaystyle \|\vec{u}\|=1\Leftrightarrow a=\frac{\pm \sqrt{3}}{2}$, $\displaystyle \|\vec{v}\|=1\Leftrightarrow b=\frac{\pm 1}{2}$ et $\vec{u}\cdot\vec{v}=0$ si, et seulement si, $a$ et $b$ sont de même signe. Enfin pour que $(\vec{u},\vec{v})$ soit directe, il faut que $\mbox{det}(\vec{u},\vec{v})$ soit positif.
\end{explanations}
\end{question}



\begin{question}
Dans le plan muni d'un repère orthonormé $(O,\vec{i},\vec{j})$, on considère les points $\displaystyle A(a,b)$ et $\displaystyle B(1,1)$. Comment choisir les réels $a$ et $b$ pour que le triangle de sommets $O,A,B$ soit rectangle et isocèle en $A$ ?
\begin{answers}  
    \bad{$\displaystyle a=-1$ et $\displaystyle b=-1$}
    \good{$\displaystyle a=1$ et $\displaystyle b=0$}
    \good{$\displaystyle a=0$ et $\displaystyle b=1$}
    \bad{$\displaystyle a=1$ et $\displaystyle b=-1$}
\end{answers}
\begin{explanations}
On doit avoir $\displaystyle \|\overrightarrow{OA}\|=\|\overrightarrow{AB}\|$ et $\overrightarrow{OA}\cdot\overrightarrow{AB}=0$. Ceci donne ($a=1$ et $b=0$) ou ($a=0$ et $b=1$).
\end{explanations}
\end{question}



\begin{question}
Soit $D$ la droite définie par l'équation cartésienne : $x-2y=4$. Quelles sont les coordonnées $(a,b)$ du projeté orthogonal $H(a,b)$ du point $M(1,1)$ sur $D$ ?
\begin{answers}  
    \bad{$(a,b)=(4,0)$}
    \good{$(a,b)=(2,-1)$}
    \bad{$(a,b)=(6,1)$}
    \bad{$(a,b)=(1,1)$}
\end{answers}
\begin{explanations}
Le vecteur $\vec{u}=(1,-2)$ est normal à $D$ et le vecteur $\vec{v}=(2,1)$ est directeur de $D$. Les coordonnées de $H$ vérifient le système
$$\left\{\begin{array}{l}a-2b=4\\ \overrightarrow{HM}\cdot\vec{v}=0
\end{array}\right. \Leftrightarrow \left\{\begin{array}{l}a-2b=4\\ 2a+b=3
\end{array}\right. \Leftrightarrow \left\{\begin{array}{l}a=2\\ b=-1.
\end{array}\right.$$
\end{explanations}
\end{question}



\begin{question}
On considère trois points $A$, $B$ et $C$ du plan tels que
$$(AB)\; : \; x-2y+3=0\quad \mbox{ et }\quad (AC)\; : \; 2x-y-3=0.$$
Quelles sont les bonnes réponses ?
\begin{answers}  
    \bad{Les points $A$, $B$ et $C$ sont alignés.}
    \bad{Le point $B$ appartient à $(AC)$.}
    \bad{Le point $C$ appartient à $(AB)$.}
    \good{Les coordonnées de $A$ sont $A(3,3)$.}
\end{answers}
\begin{explanations}
Les points $A$, $B$ et $C$ ne sont pas alignés car sinon les droites $(AB)$ et $(AC)$ seraient confondues. Ces droites se coupent en $A$ et les coordonnées de ce point d'intersection vérifient le système 
$$\left\{\begin{array}{l}x-2y+3=0\\ 2x-y-3=0
\end{array}\right. \Leftrightarrow \left\{\begin{array}{l}x=3\\ y=3.\end{array}\right.$$
\end{explanations}
\end{question}


\begin{question}
Dans le plan rapporté à un repère orthonormé, on considère le point $\displaystyle A(1,2)$ et on note $D$ une droite passant par $A$ et qui est à distance $1$ de l'origine. Une équation cartésienne de $D$ est
\begin{answers}  
    \good{$D\; :\; x=1$}
    \bad{$D\; :\; x+2y=0$}
    \good{$D\; :\; 3x-4y+5=0$}
    \bad{$D\; :\; y=2x$}
\end{answers}
\begin{explanations}
Une équation cartésienne d'une droite $D$ passant par $A$ est de la forme $a(x-1)+b(y-2)=0$. Mais,
$$1=\mbox{d}(O,D)=\frac{|a+2b|}{\sqrt{a^2+b^2}}\Leftrightarrow b=0\mbox{ ou }b=-\frac{4}{3}a.$$
Ceci détermine toutes les droites passant par $A$ et qui sont à distance $1$ de l'origine.
\end{explanations}
\end{question}


\begin{question}
Dans le plan rapporté à un repère orthonormé, on considère la droite $\Delta$ d'équation $y=x$ et on note $D$ une droite perpendiculaire à $\Delta$ et qui est à distance $1$ de l'origine. Une équation cartésienne de $D$ est
\begin{answers}  
    \bad{$D\; :\; x-y+\sqrt{2}=0$}
    \good{$D\; :\; x+y+\sqrt{2}=0$}
    \good{$D\; :\; x+y-\sqrt{2}=0$}
    \bad{$D\; :\; x-y-\sqrt{2}=0$}
\end{answers}
\begin{explanations}
Le vecteur $\vec{n}=(1,-1)$ est normal à $\Delta$, il dirige $D$. Une équation cartésienne de $D$ est de la forme $x+y+c=0$. Mais,
$$1=\mbox{d}(O,D)=\frac{|c|}{\sqrt{2}}\Leftrightarrow c=\pm \sqrt{2}.$$
\end{explanations}
\end{question}


\begin{question}
Dans le plan rapporté à un repère orthonormé, on considère la droite $\Delta$ d'équation $x=y$ et on note $D$ une droite parallèle à $\Delta$ et qui est à distance $1$ de l'origine. Une équation cartésienne de $D$ est
\begin{answers}  
    \good{$D\; :\; x-y+\sqrt{2}=0$}
    \bad{$D\; :\; x+y+\sqrt{2}=0$}
    \bad{$D\; :\; x+y-\sqrt{2}=0$}
    \good{$D\; :\; x-y-\sqrt{2}=0$}
\end{answers}
\begin{explanations}
Le vecteur $\vec{n}=(1,-1)$ est normal à $\Delta$, il est aussi normal à $D$. Une équation cartésienne de $D$ est de la forme $x-y+c=0$. Mais,
$$1=\mbox{d}(O,D)=\frac{|c|}{\sqrt{2}}\Leftrightarrow c=\pm \sqrt{2}.$$
\end{explanations}
\end{question}



\begin{question}
Dans le plan rapporté à un repère orthonormé, on considère la droite $\Delta$ d'équation $x=y$ et on note $D$ une droite perpendiculaire à $\Delta$ et qui est à distance $0$ de l'origine. Une représentation paramétrique de $D$ est
\begin{answers}  
    \bad{$D\; :\; x=t,\; y=t$ et $t\in \Rr$}
    \good{$D\; :\; x=t,\; y=-t$ et $t\in \Rr$}
    \bad{$D\; :\; x=3t,\; y=3t$ et $t\in \Rr$}
    \good{$D\; :\; x=-2t,\; y=2t$ et $t\in \Rr$}
\end{answers}
\begin{explanations}
Le vecteur $\vec{n}=(1,-1)$ est normal à $\Delta$, il dirige $D$. Or $\mbox{d}(O,D)=0\Rightarrow O\in D$. Donc $D$ est la droite passant par $O$ est dirigée par $\vec{n}$.
\end{explanations}
\end{question}


\begin{question}
Dans le plan rapporté à un repère orthonormé, on considère la droite $\Delta$ d'équation $y=x$ et on note $D$ une droite parallèle à $\Delta$ et qui est à distance $0$ de l'origine. Une représentation paramétrique de $D$ est
\begin{answers}  
    \good{$D\; :\; x=t,\; y=t$ et $t\in \Rr$}
    \bad{$D\; :\; x=t,\; y=-t$ et $t\in \Rr$}
    \good{$D\; :\; x=-t,\; y=-t$ et $t\in \Rr$}
    \bad{$D\; :\; x=2t,\; y=-2t$ et $t\in \Rr$}
\end{answers}
\begin{explanations}
Le vecteur $\vec{n}=(1,-1)$ est normal à $\Delta$, il est aussi normal à $D$. Donc $\vec{v}=(1,1)$ dirige $D$. Or $\mbox{d}(O,D)=0\Rightarrow O\in D$. Donc $D$ est la droite passant par $O$ est dirigée par $\vec{v}$.
\end{explanations}
\end{question}



\begin{question}
Le projeté orthogonal de l'origine $O$ sur une droite $D$ du plan est le point $H(1,1)$. Quelles sont les bonnes réponses ?
\begin{answers}  
    \bad{La distance entre $O$ et $D$ est $0$.}
    \good{La distance entre $O$ et $D$ est $\sqrt{2}$.}
    \good{Une équation cartésienne de $D$ est $x+y-2=0$.}
    \bad{Une équation cartésienne de $D$ est $y=x$.}
\end{answers}
\begin{explanations}
$D$ est la droite passant par $H$ et $\overrightarrow{OH}=(1,1)$ en est un vecteur normal.
\end{explanations}
\end{question}

